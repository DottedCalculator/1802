\documentclass[11pt]{article}
\usepackage{amsmath,amsthm,amssymb}
\usepackage[colorlinks]{hyperref}
\usepackage{tikz, pgfplots}
\pgfplotsset{compat=1.17}

\begin{document}
\title{Quick answer key to Recitation 16}
\author{ChatGPT 4o}
\date{4 November 2024}
\maketitle

Use the table of contents below to skip to a specific part
without seeing spoilers to the other parts.

I just used ChatGPT to write this one quickly.
ChatGPT can make mistakes, so if you spot anything that's wrong, flag me to ask.

\tableofcontents

\newpage

\section{Solution}

\subsection{(a)}

We are given the vector field $\mathbf{F}(x, y) = \langle x, -y \rangle$ on the $xy$-plane. We are to sketch this vector field and, on the same picture, draw the oriented path $C$ from $(-1, 0)$ to $(0, -1)$ given by the unit circle in the quadrant where $x \leq 0$ and $y \leq 0$.

To sketch the vector field $\mathbf{F}(x, y) = \langle x, -y \rangle$:

- At any point $(x, y)$, the vector points in the direction $\langle x, -y \rangle$.
- For example:
  - At $(1, 0)$, $\mathbf{F} = \langle 1, 0 \rangle$, pointing right.
  - At $(0, 1)$, $\mathbf{F} = \langle 0, -1 \rangle$, pointing downward.
  - At $(-1, 0)$, $\mathbf{F} = \langle -1, 0 \rangle$, pointing left.
  - At $(0, -1)$, $\mathbf{F} = \langle 0, 1 \rangle$, pointing upward.

The path $C$ is the quarter-circle from $(-1, 0)$ to $(0, -1)$ along the unit circle in the third quadrant ($x \leq 0$, $y \leq 0$). The path is oriented from $(-1, 0)$ to $(0, -1)$ in the clockwise direction.

\newpage

\subsection{(b)}

Using the picture as a guide, we can estimate whether the line integral $\displaystyle \int_C x \, dx - y \, dy$ is positive, negative, or zero.

Along the path $C$:

- Both $x$ and $y$ are negative.
- The vector field $\mathbf{F} = \langle x, -y \rangle$ evaluated along $C$ has:
  - $x$-component negative (since $x < 0$).
  - $y$-component positive (since $-y > 0$ when $y < 0$).

The tangent vector to the path $C$ is:

- Oriented clockwise from $(-1, 0)$ to $(0, -1)$.
- At each point, the tangent vector points in the direction of motion along $C$.

Since the vector field $\mathbf{F}$ and the tangent vector to $C$ are generally pointing in opposite directions (the $x$-components are both negative, but the $y$-components are opposite in sign), their dot product will be negative.

Therefore, we can expect that the line integral $\displaystyle \int_C x \, dx - y \, dy$ is \textbf{negative}.

\newpage

\subsection{Corrected Calculation for Part (c)}

Re-parametrize $C$ to match the orientation from $(-1, 0)$ to $(0, -1)$ (clockwise):

Let $\theta$ go from $\pi$ to $\tfrac{3\pi}{2}$ \textbf{decreasing}:

\[
\theta = \pi - t, \quad t \in [0, \tfrac{\pi}{2}]
\]

Then:

\[
x = \cos(\theta) = \cos(\pi - t) = -\cos t, \quad y = \sin(\theta) = \sin(\pi - t) = \sin t
\]

Compute $dx$ and $dy$:

\[
dx = \sin t \, dt, \quad dy = \cos t \, dt
\]

Compute the integral:

\[
\int_C x \, dx - y \, dy = \int_{t=0}^{\tfrac{\pi}{2}} \left( -\cos t \cdot \sin t \, dt - \sin t \cdot \cos t \, dt \right) = -2 \int_{0}^{\tfrac{\pi}{2}} \cos t \sin t \, dt
\]

Simplify using $\cos t \sin t = \tfrac{1}{2} \sin 2t$:

\[
-2 \times \tfrac{1}{2} \int_{0}^{\tfrac{\pi}{2}} \sin 2t \, dt = -\int_{0}^{\tfrac{\pi}{2}} \sin 2t \, dt
\]

Integrate:

\[
-\left( -\tfrac{1}{2} \cos 2t \right) \Big|_{0}^{\tfrac{\pi}{2}} = \tfrac{1}{2} \left( \cos \pi - \cos 0 \right) = \tfrac{1}{2} \left( -1 - 1 \right) = -1
\]

Thus, the value of the integral is $\boxed{-1}$, consistent with the result from part (d).

\newpage

\subsection{(d)}

We are to find a function $f(x, y)$ such that $\nabla f = \langle x, -y \rangle$.

Compute $f(x, y)$:

- Since $\dfrac{\partial f}{\partial x} = x$, integrate with respect to $x$:

\[
f(x, y) = \int x \, dx + g(y) = \tfrac{1}{2} x^2 + g(y)
\]

- Differentiate $f$ with respect to $y$:

\[
\dfrac{\partial f}{\partial y} = g'(y)
\]

Given that $\dfrac{\partial f}{\partial y} = -y$, we have:

\[
g'(y) = -y \implies g(y) = -\tfrac{1}{2} y^2 + C
\]

Thus, the function is:

\[
f(x, y) = \tfrac{1}{2} x^2 - \tfrac{1}{2} y^2 + C
\]

Using the fundamental theorem of line integrals:

\[
\int_C x \, dx - y \, dy = f\left( (0, -1) \right) - f\left( (-1, 0) \right)
\]

Compute $f(0, -1)$:

\[
f(0, -1) = \tfrac{1}{2} (0)^2 - \tfrac{1}{2} (-1)^2 = -\tfrac{1}{2}
\]

Compute $f(-1, 0)$:

\[
f(-1, 0) = \tfrac{1}{2} (-1)^2 - \tfrac{1}{2} (0)^2 = \tfrac{1}{2}
\]

Therefore:

\[
\int_C x \, dx - y \, dy = f(0, -1) - f(-1, 0) = \left( -\tfrac{1}{2} \right) - \left( \tfrac{1}{2} \right) = -1
\]

\newpage

\section{Solution}

\subsection{(a)}

We are asked to calculate the line integral:

\[
\int_{C} \mathbf{F} \cdot d\mathbf{r}
\]

where $\mathbf{F} = (x + y)\mathbf{i} + (x y)\mathbf{j}$, and $C$ is the broken line running from $(0, 0)$ to $(2, 2)$ to $(0, 2)$.

\newpage

\subsubsection{Parameterization of the Path $C$}

The path $C$ consists of two segments:

\begin{enumerate}
    \item Segment $C_1$: from $(0, 0)$ to $(2, 2)$.
    \item Segment $C_2$: from $(2, 2)$ to $(0, 2)$.
\end{enumerate}

\paragraph{Segment $C_1$:}

We can parametrize $C_1$ as:

\[
\begin{aligned}
x &= t, \\
y &= t, \\
t &\in [0, 2].
\end{aligned}
\]

Compute $d\mathbf{r}$:

\[
d\mathbf{r} = (dx, dy) = (dt, dt).
\]

Compute $\mathbf{F} \cdot d\mathbf{r}$:

\[
\mathbf{F} \cdot d\mathbf{r} = [(x + y), x y] \cdot (dx, dy) = (x + y) dx + x y dy.
\]

Since $x = y = t$, we have:

\[
\begin{aligned}
x + y &= t + t = 2t, \\
x y &= t \cdot t = t^2, \\
dx &= dt, \\
dy &= dt.
\end{aligned}
\]

Therefore,

\[
\mathbf{F} \cdot d\mathbf{r} = (2t)(dt) + (t^2)(dt) = [2t + t^2] dt.
\]

Compute the integral over $C_1$:

\[
\int_{C_1} \mathbf{F} \cdot d\mathbf{r} = \int_{t=0}^{2} [2t + t^2] dt = \left[ t^2 + \tfrac{1}{3} t^3 \right]_0^2 = \left( 4 + \tfrac{8}{3} \right) - 0 = \tfrac{12}{3} + \tfrac{8}{3} = \tfrac{20}{3}.
\]

\paragraph{Segment $C_2$:}

We can parametrize $C_2$ as:

\[
\begin{aligned}
x &= 2 - t, \\
y &= 2, \\
t &\in [0, 2].
\end{aligned}
\]

Compute $d\mathbf{r}$:

\[
d\mathbf{r} = (dx, dy) = (-dt, 0).
\]

Compute $\mathbf{F} \cdot d\mathbf{r}$:

\[
\mathbf{F} \cdot d\mathbf{r} = [(x + y), x y] \cdot (dx, dy) = (x + y) dx + x y dy.
\]

Since $y = 2$, $x = 2 - t$, we have:

\[
\begin{aligned}
x + y &= (2 - t) + 2 = 4 - t, \\
x y &= (2 - t)(2) = 4 - 2t, \\
dx &= -dt, \\
dy &= 0.
\end{aligned}
\]

Therefore,

\[
\mathbf{F} \cdot d\mathbf{r} = (4 - t)(-dt) + (4 - 2t)(0) = -(4 - t) dt = (-4 + t) dt.
\]

Compute the integral over $C_2$:

\[
\int_{C_2} \mathbf{F} \cdot d\mathbf{r} = \int_{t=0}^{2} (-4 + t) dt = \left[ -4t + \tfrac{1}{2} t^2 \right]_0^2 = \left( -8 + 2 \right) - 0 = -6.
\]

\paragraph{Total Integral:}

Add the integrals over $C_1$ and $C_2$:

\[
\int_{C} \mathbf{F} \cdot d\mathbf{r} = \int_{C_1} \mathbf{F} \cdot d\mathbf{r} + \int_{C_2} \mathbf{F} \cdot d\mathbf{r} = \tfrac{20}{3} - 6 = \tfrac{20}{3} - \tfrac{18}{3} = \tfrac{2}{3}.
\]

\newpage

\subsection{Answer:}

\[
\int_{C} \mathbf{F} \cdot d\mathbf{r} = \frac{2}{3}
\]

\newpage

\subsection{(b)}

We are asked to calculate the line integral:

\[
\int_{C} (x^2 - y) \, dx + 2x \, dy
\]

where $C$ is the path from $(-1, 0)$ to $(1, 0)$ along the parabola $y = 1 - x^2$.

\newpage

\subsubsection{Parametrization of the Path $C$}

We can parametrize $C$ as:

\[
\begin{aligned}
x &= t, \\
y &= 1 - t^2, \\
t &\in [-1, 1].
\end{aligned}
\]

Compute $dx$ and $dy$:

\[
dx = dt, \quad dy = -2t \, dt.
\]

Compute the integrand:

\[
(x^2 - y) \, dx + 2x \, dy.
\]

Substitute $x$, $y$, $dx$, and $dy$:

\[
\begin{aligned}
x^2 - y &= t^2 - (1 - t^2) = t^2 - 1 + t^2 = 2t^2 - 1, \\
dx &= dt, \\
2x &= 2t, \\
dy &= -2t \, dt.
\end{aligned}
\]

Compute each term:

\[
(x^2 - y) \, dx = (2t^2 - 1) dt, \quad 2x \, dy = 2t \cdot (-2t \, dt) = -4t^2 dt.
\]

Add the terms:

\[
(x^2 - y) \, dx + 2x \, dy = (2t^2 - 1) dt - 4t^2 dt = (-2t^2 - 1) dt.
\]

Simplify:

\[
(-2t^2 - 1) dt = -(2t^2 + 1) dt.
\]

Compute the integral:

\[
\int_{t=-1}^{1} -(2t^2 + 1) dt = -\left[ \tfrac{2}{3} t^3 + t \right]_{-1}^{1} = -\left( \left( \tfrac{2}{3} (1)^3 + 1 \right) - \left( \tfrac{2}{3} (-1)^3 + (-1) \right) \right).
\]

Compute the values:

\[
\begin{aligned}
\text{At } t = 1: &\quad \tfrac{2}{3} (1) + 1 = \tfrac{2}{3} + 1 = \tfrac{5}{3}, \\
\text{At } t = -1: &\quad \tfrac{2}{3} (-1) + (-1) = -\tfrac{2}{3} -1 = -\tfrac{5}{3}.
\end{aligned}
\]

Subtract:

\[
\left( \tfrac{5}{3} \right) - \left( -\tfrac{5}{3} \right) = \tfrac{10}{3}.
\]

Therefore,

\[
\int_{C} (x^2 - y) \, dx + 2x \, dy = -\left( \tfrac{10}{3} \right) = -\tfrac{10}{3}.
\]

\newpage

\subsection{Answer:}

\[
\int_{C} (x^2 - y) \, dx + 2x \, dy = -\frac{10}{3}
\]




\newpage

\section{Solution}

\subsection{Part 1: Finding the Gradient Vector Field \(\mathbf{F} = \nabla f(x, y)\)}

Given the function:
\[
f(x, y) = \sin(x) \cos(y)
\]

Compute the partial derivatives with respect to \( x \) and \( y \):

\[
\begin{aligned}
\frac{\partial f}{\partial x} &= \cos(x) \cos(y) \\
\frac{\partial f}{\partial y} &= -\sin(x) \sin(y)
\end{aligned}
\]

Therefore, the gradient vector field is:

\[
\mathbf{F} = \nabla f(x, y) = \left( \cos(x) \cos(y),\ -\sin(x) \sin(y) \right)
\]

\newpage

\subsection{Part 2: Maximizing the Line Integral \(\displaystyle \int_{C} \mathbf{F} \cdot d\mathbf{r}\)}

Since \(\mathbf{F}\) is the gradient of \( f \), the line integral over a path \( C \) from point \( A \) to point \( B \) is given by the fundamental theorem of line integrals:

\[
\int_{C} \mathbf{F} \cdot d\mathbf{r} = f(B) - f(A)
\]

To find the maximum possible value of the line integral as \( C \) ranges over all possible paths in the plane, we need to maximize the difference \( f(B) - f(A) \).

\newpage

\subsubsection{Finding the Maximum and Minimum Values of \( f(x, y) \)}

The function \( f(x, y) = \sin(x) \cos(y) \) attains its maximum and minimum values based on the ranges of the sine and cosine functions:

\[
\begin{aligned}
\sin(x) &\in [-1, 1] \\
\cos(y) &\in [-1, 1]
\end{aligned}
\]

Therefore, the maximum and minimum values of \( f(x, y) \) are:

\[
\begin{aligned}
f_{\text{max}} &= \sin(x_{\text{max}}) \cos(y_{\text{max}}) = (1)(1) = 1 \\
f_{\text{min}} &= \sin(x_{\text{min}}) \cos(y_{\text{min}}) = (-1)(-1) = 1
\end{aligned}
\]

Wait, this suggests that both the maximum and minimum values are 1, which is incorrect. Let's reconsider.

Actually, the minimum value occurs when one of the functions is 1 and the other is -1:

\[
\begin{aligned}
f_{\text{min}} &= \sin(x_{\text{min}}) \cos(y_{\text{min}}) = (1)(-1) = -1 \quad \text{or} \quad (-1)(1) = -1
\end{aligned}
\]

Therefore, the correct maximum and minimum values are:

\[
\begin{aligned}
f_{\text{max}} &= 1 \\
f_{\text{min}} &= -1
\end{aligned}
\]

\newpage

\subsubsection{Calculating the Maximum Value of the Line Integral}

The maximum possible value of the line integral is:

\[
\int_{C} \mathbf{F} \cdot d\mathbf{r} = f(B) - f(A) \leq f_{\text{max}} - f_{\text{min}} = 1 - (-1) = 2
\]

Thus, the maximum possible value of the line integral is \( \boxed{2} \).

\newpage

\subsection{Conclusion}

The gradient vector field is:

\[
\mathbf{F} = \nabla f(x, y) = \left( \cos(x) \cos(y),\ -\sin(x) \sin(y) \right)
\]

The maximum possible value of the line integral \( \displaystyle \int_{C} \mathbf{F} \cdot d\mathbf{r} \) as \( C \) ranges over all possible paths in the plane is:

\[
\int_{C} \mathbf{F} \cdot d\mathbf{r} \leq 2
\]

Therefore, the maximum value is \( \boxed{2} \).


\end{document}
