\documentclass[11pt]{article}
\usepackage{amsmath,amsthm,amssymb}
\usepackage[colorlinks]{hyperref}

\begin{document}
\title{Quick answer key to R03}
\author{ChatGPT 4o}
\date{11 September 2024}
\maketitle

Use the table of contents below to skip to a specific part
without seeing spoilers to the other parts.

I just used ChatGPT to write this one quickly.
ChatGPT can make mistakes, so if you spot anything that's wrong, flag me to ask.

\tableofcontents

\newpage

\section{Problem 1}
\subsection{Part (a)}
\textbf{Calculate $AB$ and $BA$.}

First, we compute $AB$:
\[
AB = \begin{pmatrix} 2 & 1 \\ 1 & 2 \end{pmatrix} \begin{pmatrix} -3 & 1 \\ 2 & 0 \end{pmatrix}
= \begin{pmatrix} 2(-3) + 1(2) & 2(1) + 1(0) \\ 1(-3) + 2(2) & 1(1) + 2(0) \end{pmatrix}
= \begin{pmatrix} -6 + 2 & 2 \\ -3 + 4 & 1 \end{pmatrix}
= \begin{pmatrix} -4 & 2 \\ 1 & 1 \end{pmatrix}.
\]

Now, we compute $BA$:
\[
BA = \begin{pmatrix} -3 & 1 \\ 2 & 0 \end{pmatrix} \begin{pmatrix} 2 & 1 \\ 1 & 2 \end{pmatrix}
= \begin{pmatrix} -3(2) + 1(1) & -3(1) + 1(2) \\ 2(2) + 0(1) & 2(1) + 0(2) \end{pmatrix}
= \begin{pmatrix} -6 + 1 & -3 + 2 \\ 4 + 0 & 2 + 0 \end{pmatrix}
= \begin{pmatrix} -5 & -1 \\ 4 & 2 \end{pmatrix}.
\]

\newpage

\subsection{Part (b)}
\textbf{Calculate the matrix $A^{-1}$.}

To calculate the inverse of $A$, we use the formula for a $2 \times 2$ matrix:
\[
A^{-1} = \frac{1}{\det(A)} \begin{pmatrix} d & -b \\ -c & a \end{pmatrix}, \text{ where } A = \begin{pmatrix} a & b \\ c & d \end{pmatrix}.
\]
First, compute the determinant of $A$:
\[
\det(A) = 2(2) - 1(1) = 4 - 1 = 3.
\]
Now, the inverse of $A$ is:
\[
A^{-1} = \frac{1}{3} \begin{pmatrix} 2 & -1 \\ -1 & 2 \end{pmatrix}
= \begin{pmatrix} \frac{2}{3} & -\frac{1}{3} \\ -\frac{1}{3} & \frac{2}{3} \end{pmatrix}.
\]

\newpage

\subsection{Part (c)}
\textbf{Check directly that $AA^{-1} = A^{-1}A = I$.}

We compute $AA^{-1}$:
\begin{align*}
AA^{-1} &= \begin{pmatrix} 2 & 1 \\ 1 & 2 \end{pmatrix} \begin{pmatrix} \frac{2}{3} & -\frac{1}{3} \\ -\frac{1}{3} & \frac{2}{3} \end{pmatrix} \\
&= \begin{pmatrix} 2\left(\frac{2}{3}\right) + 1\left(-\frac{1}{3}\right) & 2\left(-\frac{1}{3}\right) + 1\left(\frac{2}{3}\right) \\
1\left(\frac{2}{3}\right) + 2\left(-\frac{1}{3}\right) & 1\left(-\frac{1}{3}\right) + 2\left(\frac{2}{3}\right) \end{pmatrix} \\
&= \begin{pmatrix} \frac{4}{3} - \frac{1}{3} & -\frac{2}{3} + \frac{2}{3} \\
\frac{2}{3} - \frac{2}{3} & -\frac{1}{3} + \frac{4}{3} \end{pmatrix} \\
&= \begin{pmatrix} 1 & 0 \\ 0 & 1 \end{pmatrix} = I.
\end{align*}
Similarly, $A^{-1}A = I$.

\newpage

\subsection{Part (d)}
\textbf{Solve the system of equations using $A^{-1}$.}

The system of equations is:
\[
2x + y = 7, \quad x + 2y = 11.
\]
This can be written as the matrix equation:
\[
\begin{pmatrix} 2 & 1 \\ 1 & 2 \end{pmatrix} \begin{pmatrix} x \\ y \end{pmatrix} = \begin{pmatrix} 7 \\ 11 \end{pmatrix}.
\]
Multiplying both sides by $A^{-1}$:
\[
\begin{pmatrix} x \\ y \end{pmatrix} = A^{-1} \begin{pmatrix} 7 \\ 11 \end{pmatrix}
= \begin{pmatrix} \frac{2}{3} & -\frac{1}{3} \\ -\frac{1}{3} & \frac{2}{3} \end{pmatrix} \begin{pmatrix} 7 \\ 11 \end{pmatrix}
= \begin{pmatrix} \frac{2}{3}(7) + -\frac{1}{3}(11) \\ -\frac{1}{3}(7) + \frac{2}{3}(11) \end{pmatrix}
= \begin{pmatrix} \frac{14}{3} - \frac{11}{3} \\ -\frac{7}{3} + \frac{22}{3} \end{pmatrix}
= \begin{pmatrix} 1 \\ 5 \end{pmatrix}.
\]
So, the solution is $x = 1$, $y = 5$.

\textbf{Check the solution.}

Substituting $x = 1$ and $y = 5$ into the original equations:
\[
2(1) + 5 = 7, \quad 1 + 2(5) = 11,
\]
both of which are true, so the solution is correct.

\newpage

\section{Problem 2}

\subsection{(a) Transformation of the unit square by $T = \begin{pmatrix} 2 & 3 \\ 0 & 1 \end{pmatrix}$.}

We apply $T$ to each vertex of the unit square:
\[
T(0,0) = \begin{pmatrix} 2 & 3 \\ 0 & 1 \end{pmatrix} \begin{pmatrix} 0 \\ 0 \end{pmatrix} = \begin{pmatrix} 0 \\ 0 \end{pmatrix},
\]
\[
T(0,1) = \begin{pmatrix} 2 & 3 \\ 0 & 1 \end{pmatrix} \begin{pmatrix} 0 \\ 1 \end{pmatrix} = \begin{pmatrix} 3 \\ 1 \end{pmatrix},
\]
\[
T(1,0) = \begin{pmatrix} 2 & 3 \\ 0 & 1 \end{pmatrix} \begin{pmatrix} 1 \\ 0 \end{pmatrix} = \begin{pmatrix} 2 \\ 0 \end{pmatrix},
\]
\[
T(1,1) = \begin{pmatrix} 2 & 3 \\ 0 & 1 \end{pmatrix} \begin{pmatrix} 1 \\ 1 \end{pmatrix} = \begin{pmatrix} 5 \\ 1 \end{pmatrix}.
\]
The new vertices of the transformed square are $(0, 0)$, $(3, 1)$, $(2, 0)$, and $(5, 1)$.

\newpage

\subsection{(b) Reflection Across the Line \( y = -x \)}

The matrix \( A \) associated with reflection across the line \( y = -x \) is found by observing how the reflection changes the coordinates of a point \( (x, y) \) to \( (-y, -x) \).

For the standard basis vectors:
\[
\mathbf{e}_1 = \begin{pmatrix} 1 \\ 0 \end{pmatrix} \text{ reflects to } \begin{pmatrix} 0 \\ -1 \end{pmatrix},
\quad \mathbf{e}_2 = \begin{pmatrix} 0 \\ 1 \end{pmatrix} \text{ reflects to } \begin{pmatrix} -1 \\ 0 \end{pmatrix}.
\]
Thus, the matrix representing this reflection is:
\[
A = \begin{pmatrix} 0 & -1 \\ -1 & 0 \end{pmatrix}.
\]

\newpage

\subsection{(c) Linear Transformation in \( \mathbb{R}^3 \)}

The linear transformation \( f \) acts on vectors \( \mathbf{v} = \begin{pmatrix} x \\ y \\ z \end{pmatrix} \) by:
\[
f \begin{pmatrix} x \\ y \\ z \end{pmatrix} = \begin{pmatrix} 3x - 2z \\ x + y + z \\ 4y + z \end{pmatrix}.
\]
To find the matrix associated with this transformation, we determine how the transformation acts on the standard basis vectors \( \mathbf{e}_1 = \begin{pmatrix} 1 \\ 0 \\ 0 \end{pmatrix} \), \( \mathbf{e}_2 = \begin{pmatrix} 0 \\ 1 \\ 0 \end{pmatrix} \), and \( \mathbf{e}_3 = \begin{pmatrix} 0 \\ 0 \\ 1 \end{pmatrix} \):
\[
f(\mathbf{e}_1) = f\begin{pmatrix} 1 \\ 0 \\ 0 \end{pmatrix} = \begin{pmatrix} 3 \\ 1 \\ 0 \end{pmatrix},
\quad f(\mathbf{e}_2) = f\begin{pmatrix} 0 \\ 1 \\ 0 \end{pmatrix} = \begin{pmatrix} 0 \\ 1 \\ 4 \end{pmatrix},
\quad f(\mathbf{e}_3) = f\begin{pmatrix} 0 \\ 0 \\ 1 \end{pmatrix} = \begin{pmatrix} -2 \\ 1 \\ 1 \end{pmatrix}.
\]
Thus, the matrix associated with the transformation is:
\[
A = \begin{pmatrix}
3 & 0 & -2 \\
1 & 1 & 1 \\
0 & 4 & 1
\end{pmatrix}.
\]
\newpage

\subsection{(d) Projection onto the Vector \( \mathbf{w} = \begin{pmatrix} 1 \\ 2 \end{pmatrix} \)}
The answer is
\[
A = \begin{pmatrix} \frac{1}{5} & \frac{2}{5} \\ \frac{2}{5} & \frac{4}{5} \end{pmatrix}.
\]
This follows by showing $\mathbf{e}_1$ and $\mathbf{e}_2$ project to
$\begin{pmatrix} 1/5 \\ 2/5 \end{pmatrix}$
and $\begin{pmatrix} 2/5 \\ 4/5 \end{pmatrix}$
respectively.\footnote{ChatGPT provided a solution
    by citing a formula that I don't think was covered in the class,
    that's basically a cheat code in my opinion.
    But here it is if you want to see it.
    The projection matrix formula
    for projecting onto \( \mathbf{w} \) is:
    \[
    A = \frac{1}{\mathbf{w} \cdot \mathbf{w}} \mathbf{w} \mathbf{w}^T = \frac{1}{5} \begin{pmatrix} 1 \\ 2 \end{pmatrix} \begin{pmatrix} 1 & 2 \end{pmatrix}.
    \]
    Carrying out the matrix multiplication:
    \[
    A = \frac{1}{5} \begin{pmatrix} 1 & 2 \\ 2 & 4 \end{pmatrix} = \begin{pmatrix} \frac{1}{5} & \frac{2}{5} \\ \frac{2}{5} & \frac{4}{5} \end{pmatrix}.
    \]
    Thus, the matrix for the projection onto \( \mathbf{w} = \begin{pmatrix} 1 \\ 2 \end{pmatrix} \) is:
    \[
    A = \begin{pmatrix} \frac{1}{5} & \frac{2}{5} \\ \frac{2}{5} & \frac{4}{5} \end{pmatrix}.
    \]
}
Recall from R02 the projection of a vector \( \mathbf{v} \in \mathbb{R}^2 \) onto the vector
\( \mathbf{w} = \begin{pmatrix} 1 \\ 2 \end{pmatrix} \) is given by the formula:
\[
\frac{\mathbf{v} \cdot \mathbf{w}}{\mathbf{w} \cdot \mathbf{w}} \mathbf{w}.
\]

\end{document}
