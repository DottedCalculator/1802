\documentclass[11pt]{article}
\usepackage{amsmath,amsthm,amssymb}
\usepackage[colorlinks]{hyperref}

\begin{document}
\title{Quick answer key to Recitation 7 (Exam 1 practice)}
\author{ChatGPT 4o}
\date{September 25, 2024}
\maketitle

Use the table of contents below to skip to a specific part
without seeing spoilers to the other parts.

I just used ChatGPT to write this one quickly.
ChatGPT can make mistakes, so if you spot anything that's wrong, flag me to ask.

\tableofcontents



\newpage

\section{Solution}

We are given the points \( P = (1, 0, 1) \), \( Q = (1, 1, 2) \), and \( R = (-1, 1, 1) \). We will compute the required values step by step.

\subsection{Part 1: Vector from \( P \) to the midpoint of the line segment connecting \( Q \) and \( R \)}

The midpoint \( M \) of the line segment connecting \( Q \) and \( R \) is given by the average of the coordinates of \( Q \) and \( R \):
\[
M = \left( \frac{1 + (-1)}{2}, \frac{1 + 1}{2}, \frac{2 + 1}{2} \right) = (0, 1, \frac{3}{2})
\]

The vector connecting \( P \) to \( M \) is the difference between their coordinates:
\[
\overrightarrow{PM} = M - P = (0 - 1, 1 - 0, \frac{3}{2} - 1) = (-1, 1, \frac{1}{2})
\]

Thus, the vector from \( P \) to the midpoint of the line segment connecting \( Q \) and \( R \) is:
\[
\overrightarrow{PM} = (-1, 1, \frac{1}{2})
\]

\newpage

\subsection{Part 2: Area of the triangle with vertices \( P, Q, R \)}

The area of the triangle with vertices \( P, Q, R \) is given by:
\[
\text{Area} = \frac{1}{2} \left\| \overrightarrow{PQ} \times \overrightarrow{PR} \right\|
\]
First, we compute the vectors \( \overrightarrow{PQ} \) and \( \overrightarrow{PR} \):
\[
\overrightarrow{PQ} = Q - P = (1 - 1, 1 - 0, 2 - 1) = (0, 1, 1)
\]
\[
\overrightarrow{PR} = R - P = (-1 - 1, 1 - 0, 1 - 1) = (-2, 1, 0)
\]

Next, we compute the cross product \( \overrightarrow{PQ} \times \overrightarrow{PR} \):
\[
\overrightarrow{PQ} \times \overrightarrow{PR} = \begin{vmatrix} \mathbf{i} & \mathbf{j} & \mathbf{k} \\ 0 & 1 & 1 \\ -2 & 1 & 0 \end{vmatrix}
= \mathbf{i} \begin{vmatrix} 1 & 1 \\ 1 & 0 \end{vmatrix} - \mathbf{j} \begin{vmatrix} 0 & 1 \\ -2 & 0 \end{vmatrix} + \mathbf{k} \begin{vmatrix} 0 & 1 \\ -2 & 1 \end{vmatrix}
\]
\[
= \mathbf{i} (1(0) - 1(1)) - \mathbf{j} (0(0) - 1(-2)) + \mathbf{k} (0(1) - 1(-2))
\]
\[
= \mathbf{i}(-1) - \mathbf{j}(2) + \mathbf{k}(2)
\]
\[
= (-1, -2, 2)
\]

Now, compute the magnitude of this vector:
\[
\left\| \overrightarrow{PQ} \times \overrightarrow{PR} \right\| = \sqrt{(-1)^2 + (-2)^2 + 2^2} = \sqrt{1 + 4 + 4} = \sqrt{9} = 3
\]

Thus, the area of the triangle is:
\[
\text{Area} = \frac{1}{2} \times 3 = \frac{3}{2}
\]

\newpage

\subsection{Part 3: Equation of the plane through \( P, Q, R \)}

The normal vector to the plane is given by the cross product \( \overrightarrow{PQ} \times \overrightarrow{PR} \), which we found to be \( (-1, -2, 2) \). The equation of a plane passing through a point \( (x_0, y_0, z_0) \) with normal vector \( \mathbf{n} = \langle A, B, C \rangle \) is:
\[
A(x - x_0) + B(y - y_0) + C(z - z_0) = 0
\]
Substitute \( A = -1 \), \( B = -2 \), \( C = 2 \), and the coordinates of \( P = (1, 0, 1) \):
\[
-1(x - 1) - 2(y - 0) + 2(z - 1) = 0
\]
Simplifying:
\[
-(x - 1) - 2y + 2(z - 1) = 0
\]
\[
-x + 1 - 2y + 2z - 2 = 0
\]
\[
-x - 2y + 2z - 1 = 0
\]
Thus, the equation of the plane is:
\[
x + 2y - 2z = -1
\]




\newpage

\section{Solution}

We are given the following conditions:
\[
x_1^2 + x_2^2 + x_3^2 = 4 \quad \text{and} \quad y_1^2 + y_2^2 + y_3^2 = 9
\]
We are asked to find the range of possible values for:
\[
x_1 y_1 + x_2 y_2 + x_3 y_3
\]
This expression is the dot product of the vectors \( \mathbf{x} = \langle x_1, x_2, x_3 \rangle \) and \( \mathbf{y} = \langle y_1, y_2, y_3 \rangle \).

\subsection*{Step 1: Use the dot product formula}

The dot product of two vectors \( \mathbf{x} \) and \( \mathbf{y} \) is given by:
\[
\mathbf{x} \cdot \mathbf{y} = x_1 y_1 + x_2 y_2 + x_3 y_3 = |\mathbf{x}| |\mathbf{y}| \cos \theta
\]
where:
\[
|\mathbf{x}| = \sqrt{x_1^2 + x_2^2 + x_3^2} = \sqrt{4} = 2
\]
\[
|\mathbf{y}| = \sqrt{y_1^2 + y_2^2 + y_3^2} = \sqrt{9} = 3
\]
and \( \theta \) is the angle between the vectors \( \mathbf{x} \) and \( \mathbf{y} \).

Thus, the dot product becomes:
\[
\mathbf{x} \cdot \mathbf{y} = 2 \cdot 3 \cdot \cos \theta = 6 \cos \theta
\]

\subsection*{Step 2: Determine the range of values}

Since \( \cos \theta \) ranges between \( -1 \) and \( 1 \), the dot product \( \mathbf{x} \cdot \mathbf{y} \) will range between:
\[
6 \cos \theta \quad \text{where} \quad -1 \leq \cos \theta \leq 1
\]

Thus, the range of \( \mathbf{x} \cdot \mathbf{y} \) is:
\[
-6 \leq x_1 y_1 + x_2 y_2 + x_3 y_3 \leq 6
\]

\subsection*{Conclusion}

The range of possible values for \( x_1 y_1 + x_2 y_2 + x_3 y_3 \) is:
\[
\boxed{[-6, 6]}
\]




\newpage

\section{Solution}

We are given the planes:
\[
P_1: x + 2y + 3z = 0 \quad \text{and} \quad P_2: 2y - z = 0
\]

\subsection{Part 1: A vector parallel to both \( P_1 \) and \( P_2 \)}

To find a vector parallel to both planes, we note that the normal vector to a plane is perpendicular to all vectors lying in the plane. The normal vectors to the planes are:
\[
\mathbf{n}_1 = \langle 1, 2, 3 \rangle \quad \text{for} \quad P_1
\]
\[
\mathbf{n}_2 = \langle 0, 2, -1 \rangle \quad \text{for} \quad P_2
\]

A vector that is parallel to both planes must be perpendicular to both normal vectors. Such a vector can be found by taking the cross product of \( \mathbf{n}_1 \) and \( \mathbf{n}_2 \).

We compute the cross product \( \mathbf{n}_1 \times \mathbf{n}_2 \):
\[
\mathbf{n}_1 \times \mathbf{n}_2 = \begin{vmatrix} \mathbf{i} & \mathbf{j} & \mathbf{k} \\ 1 & 2 & 3 \\ 0 & 2 & -1 \end{vmatrix}
= \mathbf{i} \begin{vmatrix} 2 & 3 \\ 2 & -1 \end{vmatrix} - \mathbf{j} \begin{vmatrix} 1 & 3 \\ 0 & -1 \end{vmatrix} + \mathbf{k} \begin{vmatrix} 1 & 2 \\ 0 & 2 \end{vmatrix}
\]
\[
= \mathbf{i} \left( (2)(-1) - (3)(2) \right) - \mathbf{j} \left( (1)(-1) - (3)(0) \right) + \mathbf{k} \left( (1)(2) - (2)(0) \right)
\]
\[
= \mathbf{i}(-2 - 6) - \mathbf{j}(-1) + \mathbf{k}(2)
= -8\mathbf{i} + \mathbf{j} + 2\mathbf{k}
\]
\[
= \langle -8, 1, 2 \rangle
\]

Thus, a vector parallel to both planes is:
\[
\langle -8, 1, 2 \rangle
\]

\newpage

\subsection{Part 2: Distance from the point \( (2, 1, 4) \) to the plane \( P_1 \)}

The formula for the distance from a point \( (x_1, y_1, z_1) \) to a plane \( Ax + By + Cz + D = 0 \) is given by:
\[
\text{Distance} = \frac{|Ax_1 + By_1 + Cz_1 + D|}{\sqrt{A^2 + B^2 + C^2}}
\]
For the plane \( P_1: x + 2y + 3z = 0 \), we have:
\[
A = 1, \quad B = 2, \quad C = 3, \quad D = 0
\]
The point is \( (2, 1, 4) \), so we substitute the values into the formula:
\[
\text{Distance} = \frac{|1(2) + 2(1) + 3(4) + 0|}{\sqrt{1^2 + 2^2 + 3^2}}
= \frac{|2 + 2 + 12|}{\sqrt{1 + 4 + 9}} = \frac{16}{\sqrt{14}}
\]
\[
\text{Distance} = \frac{16}{\sqrt{14}} \approx 4.28
\]

Thus, the distance from the point \( (2, 1, 4) \) to the plane \( P_1 \) is \( \frac{16}{\sqrt{14}} \approx 4.28 \).




\newpage

\section{Solution}

\subsection{Part 1: Rotation matrix \( M \) associated with counterclockwise rotation by \( \frac{5\pi}{4} \)}

The matrix associated with a counterclockwise rotation by an angle \( \theta \) in \( \mathbb{R}^2 \) is given by:
\[
M = \begin{pmatrix} \cos\theta & -\sin\theta \\ \sin\theta & \cos\theta \end{pmatrix}
\]
In this case, the angle of rotation is \( \theta = \frac{5\pi}{4} \).

First, we compute \( \cos\left( \frac{5\pi}{4} \right) \) and \( \sin\left( \frac{5\pi}{4} \right) \):
\[
\cos\left( \frac{5\pi}{4} \right) = -\frac{1}{\sqrt{2}}, \quad \sin\left( \frac{5\pi}{4} \right) = -\frac{1}{\sqrt{2}}
\]

Thus, the rotation matrix \( M \) is:
\[
M = \begin{pmatrix} -\frac{1}{\sqrt{2}} & \frac{1}{\sqrt{2}} \\ -\frac{1}{\sqrt{2}} & -\frac{1}{\sqrt{2}} \end{pmatrix}
\]

\newpage

\subsection{Part 2: Calculate \( MN \) and \( NM \)}

We are given the matrix \( N = \begin{pmatrix} 1 & 2 & 4 \\ -3 & 6 & 2 \end{pmatrix} \).

\subsubsection*{Check if \( MN \) is defined}

The matrix \( M \) is a \( 2 \times 2 \) matrix, and \( N \) is a \( 2 \times 3 \) matrix. The product \( MN \) is defined because the number of columns in \( M \) matches the number of rows in \( N \). The result will be a \( 2 \times 3 \) matrix.

We now calculate \( MN \):
\[
MN = \begin{pmatrix} -\frac{1}{\sqrt{2}} & \frac{1}{\sqrt{2}} \\ -\frac{1}{\sqrt{2}} & -\frac{1}{\sqrt{2}} \end{pmatrix} \begin{pmatrix} 1 & 2 & 4 \\ -3 & 6 & 2 \end{pmatrix}
\]

We compute each element of the resulting matrix by taking the dot product of the rows of \( M \) with the columns of \( N \):
\[
MN = \begin{pmatrix}
\left( -\frac{1}{\sqrt{2}}(1) + \frac{1}{\sqrt{2}}(-3) \right) & \left( -\frac{1}{\sqrt{2}}(2) + \frac{1}{\sqrt{2}}(6) \right) & \left( -\frac{1}{\sqrt{2}}(4) + \frac{1}{\sqrt{2}}(2) \right) \\
\left( -\frac{1}{\sqrt{2}}(1) + -\frac{1}{\sqrt{2}}(-3) \right) & \left( -\frac{1}{\sqrt{2}}(2) + -\frac{1}{\sqrt{2}}(6) \right) & \left( -\frac{1}{\sqrt{2}}(4) + -\frac{1}{\sqrt{2}}(2) \right)
\end{pmatrix}
\]
\[
= \begin{pmatrix}
\frac{-1 + 3}{\sqrt{2}} & \frac{-2 + 6}{\sqrt{2}} & \frac{-4 + 2}{\sqrt{2}} \\
\frac{-1 - 3}{\sqrt{2}} & \frac{-2 - 6}{\sqrt{2}} & \frac{-4 - 2}{\sqrt{2}}
\end{pmatrix}
= \begin{pmatrix}
\frac{2}{\sqrt{2}} & \frac{4}{\sqrt{2}} & \frac{-2}{\sqrt{2}} \\
\frac{-4}{\sqrt{2}} & \frac{-8}{\sqrt{2}} & \frac{-6}{\sqrt{2}}
\end{pmatrix}
\]
\[
= \begin{pmatrix}
\sqrt{2} & 2\sqrt{2} & -\sqrt{2} \\
-2\sqrt{2} & -4\sqrt{2} & -3\sqrt{2}
\end{pmatrix}
\]

\newpage

\subsubsection*{Check if \( NM \) is defined}

The matrix \( N \) is \( 2 \times 3 \) and \( M \) is \( 2 \times 2 \). The product \( NM \) is not defined because the number of columns in \( N \) does not match the number of rows in \( M \).

Thus, \( NM \) is **not defined**.

\subsubsection*{Conclusion}

The matrix product \( MN \) is:
\[
MN = \begin{pmatrix}
\sqrt{2} & 2\sqrt{2} & -\sqrt{2} \\
-2\sqrt{2} & -4\sqrt{2} & -3\sqrt{2}
\end{pmatrix}
\]
The matrix product \( NM \) is not defined.

\newpage

\section{Solution}

We are given the vectors:
\[
\mathbf{v}_1 = \begin{pmatrix} 2 \\ 3 \\ 0 \end{pmatrix}, \quad \mathbf{v}_2 = \begin{pmatrix} 0 \\ 5 \\ 1 \end{pmatrix}, \quad \mathbf{v}_3 = \begin{pmatrix} 1 \\ 2 \\ 0 \end{pmatrix}
\]

\subsection{Part 1: Volume of the parallelepiped}
Take the determinant:
\[
  \det \begin{bmatrix} 2 & 3 & 0 \\ 0 & 5 & 1 \\ 1 & 2 & 0 \end{bmatrix} = -1.
\]

Thus, the volume of the parallelepiped is:
\[
\text{Volume} = | -1 | = 1
\]

\newpage

\subsection{Part 2: Are these vectors a basis for \( \mathbb{R}^3 \)?}

To determine if the vectors \( \mathbf{v}_1 \), \( \mathbf{v}_2 \), and \( \mathbf{v}_3 \) form a basis for \( \mathbb{R}^3 \), we check if they are linearly independent.
The vectors are linearly independent if the determinant above was non-zero.

Since the determinant is \( -1 \), which is non-zero, the vectors are linearly independent.

Therefore, the vectors \( \mathbf{v}_1 \), \( \mathbf{v}_2 \), and \( \mathbf{v}_3 \) form a basis for \( \mathbb{R}^3 \).




\newpage



\newpage

\section{Solution}

We are given the system of equations:
\[
x + 3y = 0 \quad \text{(1)}
\]
\[
-ax - y = 1 \quad \text{(2)}
\]

\subsection{Part 1: Matrix equation}

We can write this system of equations as a matrix equation. The system can be written as:
\[
\begin{pmatrix} 1 & 3 \\ -a & -1 \end{pmatrix} \begin{pmatrix} x \\ y \end{pmatrix} = \begin{pmatrix} 0 \\ 1 \end{pmatrix}
\]

Thus, the matrix equation is:
\[
A \mathbf{v} = \mathbf{b}
\]
where:
\[
A = \begin{pmatrix} 1 & 3 \\ -a & -1 \end{pmatrix}, \quad \mathbf{v} = \begin{pmatrix} x \\ y \end{pmatrix}, \quad \mathbf{b} = \begin{pmatrix} 0 \\ 1 \end{pmatrix}
\]

\newpage

\subsection{Part 2: Values of \( a \) for which the system has a unique solution}

The system has a unique solution if the matrix \( A \) is invertible, which occurs when the determinant of \( A \) is non-zero. The determinant of \( A \) is given by:
\[
\det(A) = \det\begin{pmatrix} 1 & 3 \\ -a & -1 \end{pmatrix} = (1)(-1) - (3)(-a) = -1 + 3a
\]

For the matrix to be invertible, we require \( \det(A) \neq 0 \):
\[
-1 + 3a \neq 0
\]
\[
3a \neq 1 \quad \Rightarrow \quad a \neq \frac{1}{3}
\]

Thus, the system has a unique solution for all values of \( a \) except \( a = \frac{1}{3} \).

\newpage

\subsection{Part 3: Solution in terms of \( a \) using the inverse matrix}

To solve the system using the inverse of the matrix, we first compute the inverse of \( A \), assuming \( a \neq \frac{1}{3} \).

The inverse of a \( 2 \times 2 \) matrix \( A = \begin{pmatrix} a & b \\ c & d \end{pmatrix} \) is given by:
\[
A^{-1} = \frac{1}{\det(A)} \begin{pmatrix} d & -b \\ -c & a \end{pmatrix}
\]

For the matrix \( A = \begin{pmatrix} 1 & 3 \\ -a & -1 \end{pmatrix} \), we already know that:
\[
\det(A) = -1 + 3a
\]

Thus, the inverse of \( A \) is:
\[
A^{-1} = \frac{1}{-1 + 3a} \begin{pmatrix} -1 & -3 \\ a & 1 \end{pmatrix}
\]

Now, we solve for \( \mathbf{v} = A^{-1} \mathbf{b} \):
\[
\mathbf{v} = \frac{1}{-1 + 3a} \begin{pmatrix} -1 & -3 \\ a & 1 \end{pmatrix} \begin{pmatrix} 0 \\ 1 \end{pmatrix}
\]
\[
= \frac{1}{-1 + 3a} \begin{pmatrix} -3 \\ 1 \end{pmatrix}
\]

Thus, the solution is:
\[
\mathbf{v} = \begin{pmatrix} x \\ y \end{pmatrix} = \frac{1}{-1 + 3a} \begin{pmatrix} -3 \\ 1 \end{pmatrix}
\]

The solutions for \( x \) and \( y \) in terms of \( a \) are:
\[
x = \frac{-3}{-1 + 3a}, \quad y = \frac{1}{-1 + 3a}
\]




\newpage

\section{Solution}

We are given the matrix:
\[
A = \begin{pmatrix} 5 & 8 \\ 7 & 4 \end{pmatrix}
\]

\subsection{Part 1: Characteristic polynomial and eigenvalues}

The characteristic polynomial of a matrix \( A \) is given by:
\[
\det(A - \lambda I) = 0
\]
where \( \lambda \) is an eigenvalue and \( I \) is the identity matrix.

We compute \( A - \lambda I \):
\[
A - \lambda I = \begin{pmatrix} 5 & 8 \\ 7 & 4 \end{pmatrix} - \lambda \begin{pmatrix} 1 & 0 \\ 0 & 1 \end{pmatrix} = \begin{pmatrix} 5 - \lambda & 8 \\ 7 & 4 - \lambda \end{pmatrix}
\]

Now, we compute the determinant of \( A - \lambda I \):
\[
\det(A - \lambda I) = \det\begin{pmatrix} 5 - \lambda & 8 \\ 7 & 4 - \lambda \end{pmatrix}
\]
\[
= (5 - \lambda)(4 - \lambda) - (8)(7)
\]
\[
= (5 - \lambda)(4 - \lambda) - 56
\]
\[
= 20 - 9\lambda + \lambda^2 - 56 = \lambda^2 - 9\lambda - 36
\]

Thus, the characteristic polynomial is:
\[
\lambda^2 - 9\lambda - 36 = 0
\]

We solve this quadratic equation using the quadratic formula:
\[
\lambda = \frac{-(-9) \pm \sqrt{(-9)^2 - 4(1)(-36)}}{2(1)} = \frac{9 \pm \sqrt{81 + 144}}{2} = \frac{9 \pm \sqrt{225}}{2} = \frac{9 \pm 15}{2}
\]

The solutions are:
\[
\lambda_1 = \frac{9 + 15}{2} = 12, \quad \lambda_2 = \frac{9 - 15}{2} = -3
\]

Thus, the eigenvalues of \( A \) are:
\[
\lambda_1 = 12, \quad \lambda_2 = -3
\]

\newpage

\subsection{Part 2: Eigenvector corresponding to the largest eigenvalue \( \lambda_1 = 12 \)}

To find the eigenvector corresponding to \( \lambda_1 = 12 \), we solve the system:
\[
(A - 12I) \mathbf{v} = 0
\]
where \( \mathbf{v} = \begin{pmatrix} v_1 \\ v_2 \end{pmatrix} \).

First, compute \( A - 12I \):
\[
A - 12I = \begin{pmatrix} 5 - 12 & 8 \\ 7 & 4 - 12 \end{pmatrix} = \begin{pmatrix} -7 & 8 \\ 7 & -8 \end{pmatrix}
\]

Now, solve the system \( (A - 12I) \mathbf{v} = 0 \):
\[
\begin{pmatrix} -7 & 8 \\ 7 & -8 \end{pmatrix} \begin{pmatrix} v_1 \\ v_2 \end{pmatrix} = \begin{pmatrix} 0 \\ 0 \end{pmatrix}
\]

This gives the system of equations:
\[
-7v_1 + 8v_2 = 0 \quad \text{(1)}
\]
\[
7v_1 - 8v_2 = 0 \quad \text{(2)}
\]

Both equations are the same, so we can solve for \( v_1 \) in terms of \( v_2 \). From equation (1), we get:
\[
-7v_1 + 8v_2 = 0 \quad \Rightarrow \quad v_1 = \frac{8}{7}v_2
\]

Thus, a corresponding eigenvector is:
\[
\mathbf{v} = \begin{pmatrix} \frac{8}{7}v_2 \\ v_2 \end{pmatrix} = v_2 \begin{pmatrix} \frac{8}{7} \\ 1 \end{pmatrix}
\]

For simplicity, we can choose \( v_2 = 7 \), which gives:
\[
\mathbf{v} = \begin{pmatrix} 8 \\ 7 \end{pmatrix}
\]

Therefore, an eigenvector corresponding to \( \lambda_1 = 12 \) is:
\[
\mathbf{v} = \begin{pmatrix} 8 \\ 7 \end{pmatrix}
\]




\newpage

\section{Solution}

\subsection{Part 1: Calculate \( (1 - i\sqrt{3})^7 \) by converting to polar form}

We are tasked with finding \( (1 - i\sqrt{3})^7 \). To do this, we first convert \( 1 - i\sqrt{3} \) to polar form and then apply De Moivre's theorem.

\paragraph{Step 1: Convert to polar form}

The modulus \( r \) of \( 1 - i\sqrt{3} \) is:
\[
r = |1 - i\sqrt{3}| = \sqrt{1^2 + (-\sqrt{3})^2} = \sqrt{1 + 3} = \sqrt{4} = 2
\]

Next, we calculate the argument \( \theta \):
\[
\theta = \arg(1 - i\sqrt{3}) = \tan^{-1}\left( \frac{-\sqrt{3}}{1} \right) = -\frac{\pi}{3}
\]
Thus, the polar form of \( 1 - i\sqrt{3} \) is:
\[
1 - i\sqrt{3} = 2 \left( \cos\left( -\frac{\pi}{3} \right) + i \sin\left( -\frac{\pi}{3} \right) \right)
\]

\paragraph{Step 2: Apply De Moivre's theorem}

Using De Moivre's theorem, we compute:
\[
(1 - i\sqrt{3})^7 = \left[ 2 \left( \cos\left( -\frac{\pi}{3} \right) + i \sin\left( -\frac{\pi}{3} \right) \right) \right]^7
\]
\[
= 2^7 \left( \cos\left( 7 \times -\frac{\pi}{3} \right) + i \sin\left( 7 \times -\frac{\pi}{3} \right) \right)
\]
\[
= 128 \left( \cos\left( -\frac{7\pi}{3} \right) + i \sin\left( -\frac{7\pi}{3} \right) \right)
\]

Since \( -\frac{7\pi}{3} = -2\pi - \frac{\pi}{3} \), we simplify using periodicity:
\[
\cos\left( -\frac{7\pi}{3} \right) = \cos\left( -\frac{\pi}{3} \right) = \frac{1}{2}, \quad \sin\left( -\frac{7\pi}{3} \right) = \sin\left( -\frac{\pi}{3} \right) = -\frac{\sqrt{3}}{2}
\]

Thus, we have:
\[
(1 - i\sqrt{3})^7 = 128 \left( \frac{1}{2} + i \left( -\frac{\sqrt{3}}{2} \right) \right)
\]
\[
= 128 \times \frac{1}{2} + 128 \times \left( -\frac{\sqrt{3}}{2} \right) i
\]
\[
= 64 - 64\sqrt{3} i
\]

Therefore:
\[
(1 - i\sqrt{3})^7 = 64 - 64\sqrt{3} i
\]

\newpage

\subsection{Part 2: Find \( zw \) and \( \frac{z}{\overline{w}} \)}

We are given \( z = 2 + 3i \) and \( w = 1 + 2i \).

\paragraph{Step 1: Compute \( zw \)}

The product of two complex numbers is given by:
\[
zw = (2 + 3i)(1 + 2i)
\]
Expand the product:
\[
zw = 2(1) + 2(2i) + 3i(1) + 3i(2i)
\]
\[
= 2 + 4i + 3i + 6i^2
\]
Since \( i^2 = -1 \), this simplifies to:
\[
zw = 2 + 7i + 6(-1) = 2 + 7i - 6 = -4 + 7i
\]

Thus:
\[
zw = -4 + 7i
\]

\paragraph{Step 2: Compute \( \frac{z}{\overline{w}} \)}

The conjugate of \( w = 1 + 2i \) is \( \overline{w} = 1 - 2i \). We now compute \( \frac{z}{\overline{w}} \):
\[
\frac{z}{\overline{w}} = \frac{2 + 3i}{1 - 2i}
\]
We multiply the numerator and denominator by the conjugate of the denominator \( 1 + 2i \):
\[
\frac{z}{\overline{w}} = \frac{(2 + 3i)(1 + 2i)}{(1 - 2i)(1 + 2i)}
\]

First, compute the denominator:
\[
(1 - 2i)(1 + 2i) = 1^2 - (2i)^2 = 1 - (-4) = 5
\]

Now, compute the numerator:
\[
(2 + 3i)(1 + 2i) = 2(1) + 2(2i) + 3i(1) + 3i(2i)
\]
\[
= 2 + 4i + 3i + 6i^2 = 2 + 7i + 6(-1) = 2 + 7i - 6 = -4 + 7i
\]

Thus:
\[
\frac{z}{\overline{w}} = \frac{-4 + 7i}{5} = -\frac{4}{5} + \frac{7}{5}i
\]

Therefore:
\[
\frac{z}{\overline{w}} = -\frac{4}{5} + \frac{7}{5}i
\]


\end{document}
