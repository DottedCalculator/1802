\documentclass[11pt]{article}
\usepackage{amsmath,amsthm,amssymb}
\usepackage[colorlinks]{hyperref}
\usepackage{tikz, pgfplots}
\pgfplotsset{compat=1.17}

\begin{document}
\title{Quick answer key to Recitation 23}
\author{ChatGPT 4o}
\date{9 December 2024}
\maketitle

Use the table of contents below to skip to a specific part
without seeing spoilers to the other parts.

I just used ChatGPT to write this one quickly.
ChatGPT can make mistakes, so if you spot anything that's wrong, flag me to ask.

\tableofcontents



\newpage

\section{Computing the Curl of a Vector Field}

We are tasked with finding the curl of the vector field:
\[
\mathbf{V}(x, y, z) = \langle x^2 y, \ yz, \ x y z^2 \rangle
\]

\newpage

\subsection{Definition of Curl}

The curl of a vector field \( \mathbf{V} = \langle P, Q, R \rangle \) is given by:
\[
\nabla \times \mathbf{V} = \left( \frac{\partial R}{\partial y} - \frac{\partial Q}{\partial z} \right) \mathbf{i} - \left( \frac{\partial R}{\partial x} - \frac{\partial P}{\partial z} \right) \mathbf{j} + \left( \frac{\partial Q}{\partial x} - \frac{\partial P}{\partial y} \right) \mathbf{k}
\]
Alternatively, using the determinant form:
\[
\nabla \times \mathbf{V} = 
\begin{vmatrix}
\mathbf{i} & \mathbf{j} & \mathbf{k} \\
\frac{\partial}{\partial x} & \frac{\partial}{\partial y} & \frac{\partial}{\partial z} \\
P & Q & R \\
\end{vmatrix}
\]

\newpage

\subsection{Components of the Vector Field}

Given:
\[
P = x^2 y, \quad Q = y z, \quad R = x y z^2
\]

\newpage

\subsection{Calculating the Partial Derivatives}

\paragraph{Component \( \mathbf{i} \):}
\[
\frac{\partial R}{\partial y} = \frac{\partial}{\partial y} (x y z^2) = x z^2
\]
\[
\frac{\partial Q}{\partial z} = \frac{\partial}{\partial z} (y z) = y
\]
\[
\left( \frac{\partial R}{\partial y} - \frac{\partial Q}{\partial z} \right) = x z^2 - y
\]

\paragraph{Component \( \mathbf{j} \):}
\[
\frac{\partial R}{\partial x} = \frac{\partial}{\partial x} (x y z^2) = y z^2
\]
\[
\frac{\partial P}{\partial z} = \frac{\partial}{\partial z} (x^2 y) = 0
\]
\[
\left( \frac{\partial R}{\partial x} - \frac{\partial P}{\partial z} \right) = y z^2 - 0 = y z^2
\]
\[
- \left( \frac{\partial R}{\partial x} - \frac{\partial P}{\partial z} \right) = - y z^2
\]

\paragraph{Component \( \mathbf{k} \):}
\[
\frac{\partial Q}{\partial x} = \frac{\partial}{\partial x} (y z) = 0
\]
\[
\frac{\partial P}{\partial y} = \frac{\partial}{\partial y} (x^2 y) = x^2
\]
\[
\left( \frac{\partial Q}{\partial x} - \frac{\partial P}{\partial y} \right) = 0 - x^2 = -x^2
\]

\newpage

\subsection{Assembling the Curl}

Combining the components, we obtain:
\[
\nabla \times \mathbf{V} = \langle x z^2 - y, \ - y z^2, \ -x^2 \rangle
\]

\newpage

\subsection{Final Answer}

The curl of the vector field \( \mathbf{V}(x, y, z) = \langle x^2 y, \ yz, \ x y z^2 \rangle \) is:
\[
\boxed{ \nabla \times \mathbf{V} = \langle x z^2 - y,\ - y z^2,\ -x^2 \rangle }
\]




\newpage

\section{Conservative Vector Fields and Line Integrals}

We are given the vector field:
\[
\mathbf{F}(x, y, z) = \langle y z^2, \ x z^2 + a y z, \ b x y z + y^2 \rangle
\]
and a parametrized curve:
\[
\mathbf{C}(t) = \left( e^{t^2 - t} - 1, \ t^4, \ \sin(\pi t) \right), \quad 0 \leq t \leq 1
\]
We aim to determine the values of \( a \) and \( b \) for which \( \mathbf{F} \) is a conservative vector field, find a corresponding potential function \( f(x, y, z) \), and compute the line integral \( \int_{C} \mathbf{F} \cdot d\mathbf{r} \) using the fundamental theorem of calculus for line integrals.

\newpage

\subsection{Part (a): Determining Values of \( a \) and \( b \) for Conservativity}

A vector field \( \mathbf{F} \) is conservative if there exists a scalar potential function \( f \) such that \( \mathbf{F} = \nabla f \). A necessary condition for conservativity in simply connected domains is that the curl of \( \mathbf{F} \) is zero everywhere:
\[
\nabla \times \mathbf{F} = \mathbf{0}
\]

\subsubsection*{Computing the Curl of \( \mathbf{F} \)}

Given:
\[
\mathbf{F} = \langle y z^2, \ x z^2 + a y z, \ b x y z + y^2 \rangle
\]
The curl of \( \mathbf{F} \) is:
\[
\nabla \times \mathbf{F} = \left( \frac{\partial R}{\partial y} - \frac{\partial Q}{\partial z} \right) \mathbf{i} - \left( \frac{\partial R}{\partial x} - \frac{\partial P}{\partial z} \right) \mathbf{j} + \left( \frac{\partial Q}{\partial x} - \frac{\partial P}{\partial y} \right) \mathbf{k}
\]
Where:
\[
P = y z^2, \quad Q = x z^2 + a y z, \quad R = b x y z + y^2
\]

Calculating each component:

\paragraph{Component \( \mathbf{i} \):}
\[
\frac{\partial R}{\partial y} = \frac{\partial}{\partial y} (b x y z + y^2) = b x z + 2y
\]
\[
\frac{\partial Q}{\partial z} = \frac{\partial}{\partial z} (x z^2 + a y z) = 2x z + a y
\]
\[
\text{Curl}_x = \frac{\partial R}{\partial y} - \frac{\partial Q}{\partial z} = (b x z + 2y) - (2x z + a y) = (b - 2)x z + (2 - a) y
\]

\paragraph{Component \( \mathbf{j} \):}
\[
\frac{\partial R}{\partial x} = \frac{\partial}{\partial x} (b x y z + y^2) = b y z
\]
\[
\frac{\partial P}{\partial z} = \frac{\partial}{\partial z} (y z^2) = 2 y z
\]
\[
\text{Curl}_y = -\left( \frac{\partial R}{\partial x} - \frac{\partial P}{\partial z} \right) = - (b y z - 2 y z) = - (b - 2) y z
\]

\paragraph{Component \( \mathbf{k} \):}
\[
\frac{\partial Q}{\partial x} = \frac{\partial}{\partial x} (x z^2 + a y z) = z^2
\]
\[
\frac{\partial P}{\partial y} = \frac{\partial}{\partial y} (y z^2) = z^2
\]
\[
\text{Curl}_z = \frac{\partial Q}{\partial x} - \frac{\partial P}{\partial y} = z^2 - z^2 = 0
\]

Thus, the curl of \( \mathbf{F} \) is:
\[
\nabla \times \mathbf{F} = \left[ (b - 2)x z + (2 - a) y \right] \mathbf{i} - (b - 2) y z \mathbf{j} + 0 \mathbf{k}
\]

\subsubsection*{Setting the Curl to Zero}

For \( \mathbf{F} \) to be conservative, \( \nabla \times \mathbf{F} = \mathbf{0} \) for all \( x, y, z \). Therefore:
\[
(b - 2)x z + (2 - a) y = 0 \quad \text{and} \quad - (b - 2) y z = 0
\]
for all \( x, y, z \).

From the second equation:
\[
- (b - 2) y z = 0
\]
Since this must hold for all \( y \) and \( z \), we must have:
\[
b - 2 = 0 \quad \Rightarrow \quad b = 2
\]

Substituting \( b = 2 \) into the first equation:
\[
(2 - 2) x z + (2 - a) y = 0 \quad \Rightarrow \quad (2 - a) y = 0
\]
For this to hold for all \( y \), we must have:
\[
2 - a = 0 \quad \Rightarrow \quad a = 2
\]

\newpage

\subsection{Part (a) Answer}

The vector field \( \mathbf{F} = \langle y z^2, \ x z^2 + a y z, \ b x y z + y^2 \rangle \) is conservative if and only if:
\[
\boxed{ a = 2 \quad \text{and} \quad b = 2 }
\]

\newpage

\subsection{Part (b): Finding the Potential Function \( f(x, y, z) \)}

Given \( a = 2 \) and \( b = 2 \), the vector field becomes:
\[
\mathbf{F} = \langle y z^2, \ x z^2 + 2 y z, \ 2 x y z + y^2 \rangle
\]
We seek a scalar function \( f(x, y, z) \) such that:
\[
\mathbf{F} = \nabla f = \left\langle \frac{\partial f}{\partial x}, \ \frac{\partial f}{\partial y}, \ \frac{\partial f}{\partial z} \right\rangle
\]

\subsubsection*{Integrating \( \frac{\partial f}{\partial x} = y z^2 \) with respect to \( x \)}
\[
f(x, y, z) = \int y z^2 \, dx + g(y, z) = y z^2 x + g(y, z)
\]

\subsubsection*{Determining \( g(y, z) \) by Differentiating with Respect to \( y \)}
\[
\frac{\partial f}{\partial y} = z^2 x + \frac{\partial g}{\partial y} = x z^2 + 2 y z
\]
\[
\Rightarrow \frac{\partial g}{\partial y} = 2 y z
\]
\[
\Rightarrow g(y, z) = \int 2 y z \, dy + h(z) = y^2 z + h(z)
\]

\subsubsection*{Updating \( f(x, y, z) \)}
\[
f(x, y, z) = y z^2 x + y^2 z + h(z)
\]

\subsubsection*{Determining \( h(z) \) by Differentiating with Respect to \( z \)}
\[
\frac{\partial f}{\partial z} = 2 y z x + y^2 + \frac{dh}{dz} = 2 x y z + y^2
\]
\[
\Rightarrow 2 x y z + y^2 + \frac{dh}{dz} = 2 x y z + y^2
\]
\[
\Rightarrow \frac{dh}{dz} = 0
\]
\[
\Rightarrow h(z) = \text{constant} = C
\]
For simplicity, set \( C = 0 \).

\subsubsection*{Final Potential Function}
\[
\boxed{ f(x, y, z) = x y z^2 + y^2 z }
\]

\newpage

\subsection{Part (c): Calculating the Line Integral Using the Potential Function}

Given the parametrized curve:
\[
\mathbf{C}(t) = \left( e^{t^2 - t} - 1, \ t^4, \ \sin(\pi t) \right), \quad 0 \leq t \leq 1
\]
Since \( \mathbf{F} \) is conservative and \( \mathbf{F} = \nabla f \), the fundamental theorem for line integrals states:
\[
\int_{C} \mathbf{F} \cdot d\mathbf{r} = f(\mathbf{C}(1)) - f(\mathbf{C}(0))
\]

\subsubsection*{Determining the Endpoints}
\[
\mathbf{C}(0) = \left( e^{0 - 0} - 1, \ 0^4, \ \sin(0) \right) = (1 - 1, \ 0, \ 0) = (0, 0, 0)
\]
\[
\mathbf{C}(1) = \left( e^{1 - 1} - 1, \ 1^4, \ \sin(\pi \cdot 1) \right) = (1 - 1, \ 1, \ 0) = (0, 1, 0)
\]

\subsubsection*{Evaluating the Potential Function at the Endpoints}
\[
f(0, 0, 0) = 0 \cdot 0 \cdot 0^2 + 0^2 \cdot 0 = 0 + 0 = 0
\]
\[
f(0, 1, 0) = 0 \cdot 1 \cdot 0^2 + 1^2 \cdot 0 = 0 + 0 = 0
\]

\subsubsection*{Calculating the Line Integral}
\[
\int_{C} \mathbf{F} \cdot d\mathbf{r} = f(0, 1, 0) - f(0, 0, 0) = 0 - 0 = 0
\]

\newpage

\subsection{Final Answer}

\begin{enumerate}
    \item[(a)] The vector field \( \mathbf{F} = \langle y z^2, \ x z^2 + a y z, \ b x y z + y^2 \rangle \) is conservative if and only if:
    \[
    \boxed{ a = 2 \quad \text{and} \quad b = 2 }
    \]
    
    \item[(b)] For \( a = 2 \) and \( b = 2 \), a corresponding potential function is:
    \[
    \boxed{ f(x, y, z) = x y z^2 + y^2 z }
    \]
    
    \item[(c)] The line integral of \( \mathbf{F} \) along the curve \( C \) is:
    \[
    \boxed{ \int_{C} \mathbf{F} \cdot d\mathbf{r} = 0 }
    \]
\end{enumerate}




\newpage

\section{Verification of Stokes' Theorem}

We aim to verify Stokes' Theorem for the following scenario:

\begin{itemize}
    \item \textbf{Surface \( S \):} The upper hemisphere of the unit sphere centered at the origin, defined by \( x^2 + y^2 + z^2 = 1 \) and \( z \geq 0 \).
    \item \textbf{Curve \( C \):} The boundary of \( S \), which is the unit circle in the \( xy \)-plane, defined by \( x^2 + y^2 = 1 \) and \( z = 0 \).
    \item \textbf{Vector Field \( \mathbf{F} \):} \( \mathbf{F}(x, y, z) = \langle x, y, z \rangle \).
\end{itemize}

Stokes' Theorem states that:
\[
\iint_{S} (\nabla \times \mathbf{F}) \cdot \mathbf{n} \, dS = \oint_{C} \mathbf{F} \cdot d\mathbf{r}
\]
where:
\begin{itemize}
    \item \( \nabla \times \mathbf{F} \) is the curl of \( \mathbf{F} \).
    \item \( \mathbf{n} \) is the unit normal vector to the surface \( S \).
    \item \( dS \) is the differential element of the surface area.
    \item \( d\mathbf{r} \) is the differential element of the curve \( C \).
\end{itemize}

Our goal is to compute both the surface integral of the curl of \( \mathbf{F} \) over \( S \) and the line integral of \( \mathbf{F} \) around \( C \), and verify that they are equal.

\newpage

\subsection{Step 1: Compute the Curl of \( \mathbf{F} \)}

Given the vector field:
\[
\mathbf{F}(x, y, z) = \langle x, y, z \rangle
\]
The curl of \( \mathbf{F} \) is computed as:
\[
\nabla \times \mathbf{F} = \left\langle \frac{\partial F_z}{\partial y} - \frac{\partial F_y}{\partial z}, \frac{\partial F_x}{\partial z} - \frac{\partial F_z}{\partial x}, \frac{\partial F_y}{\partial x} - \frac{\partial F_x}{\partial y} \right\rangle
\]
Substituting the components of \( \mathbf{F} \):
\[
\frac{\partial F_z}{\partial y} = \frac{\partial z}{\partial y} = 0, \quad \frac{\partial F_y}{\partial z} = \frac{\partial y}{\partial z} = 0
\]
\[
\frac{\partial F_x}{\partial z} = \frac{\partial x}{\partial z} = 0, \quad \frac{\partial F_z}{\partial x} = \frac{\partial z}{\partial x} = 0
\]
\[
\frac{\partial F_y}{\partial x} = \frac{\partial y}{\partial x} = 0, \quad \frac{\partial F_x}{\partial y} = \frac{\partial x}{\partial y} = 0
\]
Thus, the curl simplifies to:
\[
\nabla \times \mathbf{F} = \langle 0 - 0, \ 0 - 0, \ 0 - 0 \rangle = \langle 0, 0, 0 \rangle
\]
\[
\boxed{ \nabla \times \mathbf{F} = \mathbf{0} }
\]

\newpage

\subsection{Step 2: Compute the Surface Integral of the Curl}

Given that \( \nabla \times \mathbf{F} = \mathbf{0} \), the surface integral becomes:
\[
\iint_{S} (\nabla \times \mathbf{F}) \cdot \mathbf{n} \, dS = \iint_{S} \mathbf{0} \cdot \mathbf{n} \, dS = 0
\]
\[
\boxed{ \iint_{S} (\nabla \times \mathbf{F}) \cdot \mathbf{n} \, dS = 0 }
\]

\newpage

\subsection{Step 3: Compute the Line Integral}

To compute the line integral \( \oint_{C} \mathbf{F} \cdot d\mathbf{r} \), we parametrize the curve \( C \).

\paragraph{Parametrization of \( C \):}
Since \( C \) is the unit circle in the \( xy \)-plane, we can parametrize it using the parameter \( t \) as follows:
\[
\mathbf{r}(t) = \langle \cos t, \sin t, 0 \rangle, \quad 0 \leq t \leq 2\pi
\]
\[
\mathbf{r}'(t) = \langle -\sin t, \cos t, 0 \rangle
\]

\paragraph{Evaluating \( \mathbf{F} \) Along \( C \):}
\[
\mathbf{F}(\mathbf{r}(t)) = \langle \cos t, \sin t, 0 \rangle
\]

\paragraph{Dot Product \( \mathbf{F} \cdot \mathbf{r}'(t) \):}
\[
\mathbf{F}(\mathbf{r}(t)) \cdot \mathbf{r}'(t) = \langle \cos t, \sin t, 0 \rangle \cdot \langle -\sin t, \cos t, 0 \rangle = -\cos t \sin t + \sin t \cos t + 0 = 0
\]

\paragraph{Line Integral:}
\[
\oint_{C} \mathbf{F} \cdot d\mathbf{r} = \int_{0}^{2\pi} 0 \, dt = 0
\]
\[
\boxed{ \oint_{C} \mathbf{F} \cdot d\mathbf{r} = 0 }
\]

\newpage

\subsection{Conclusion}

Both the surface integral of the curl of \( \mathbf{F} \) over \( S \) and the line integral of \( \mathbf{F} \) around \( C \) are zero. Therefore, Stokes' Theorem holds for this vector field and surface.

\[
\boxed{ \iint_{S} (\nabla \times \mathbf{F}) \cdot \mathbf{n} \, dS = \oint_{C} \mathbf{F} \cdot d\mathbf{r} = 0 }
\]


\end{document}
