\documentclass[11pt]{article}
\usepackage{amsmath,amsthm,amssymb}
\usepackage[colorlinks]{hyperref}
\usepackage{tikz, pgfplots}
\pgfplotsset{compat=1.17}

\begin{document}
\title{Quick answer key to Recitation 15}
\author{ChatGPT 4o}
\date{Wednesday, 30 October 2024}
\maketitle

Use the table of contents below to skip to a specific part
without seeing spoilers to the other parts.

I just used ChatGPT to write this one quickly.
ChatGPT can make mistakes, so if you spot anything that's wrong, flag me to ask.

\tableofcontents



We define the parallelogram coordinates \( (u, v) \) by the transformations \( x = u + v \) and \( y = u - v \).

\newpage

\section{Problem Solution}

\begin{enumerate}
    \item \textbf{Sketch the unit square in \( (u, v) \)-coordinates on the \( xy \)-plane.}

    To map the unit square with vertices \( (0, 0), (1, 0), (0, 1), (1, 1) \) in \( (u, v) \)-coordinates onto the \( xy \)-plane, we calculate the coordinates of each vertex under the transformations \( x = u + v \) and \( y = u - v \):
    \begin{itemize}
        \item \( (u, v) = (0, 0): \quad (x, y) = (0 + 0, 0 - 0) = (0, 0) \).
        \item \( (u, v) = (1, 0): \quad (x, y) = (1 + 0, 1 - 0) = (1, 1) \).
        \item \( (u, v) = (0, 1): \quad (x, y) = (0 + 1, 0 - 1) = (1, -1) \).
        \item \( (u, v) = (1, 1): \quad (x, y) = (1 + 1, 1 - 1) = (2, 0) \).
    \end{itemize}
    Therefore, the unit square in \( (u, v) \)-coordinates maps to a parallelogram in \( (x, y) \)-coordinates with vertices at \( (0, 0), (1, 1), (1, -1), \) and \( (2, 0) \).

    \item \textbf{Evaluate \( dxdy \) in terms of \( dudv \).}

    To find \( dx \, dy \) in terms of \( du \, dv \), we compute the Jacobian determinant:
    \[
    \frac{\partial(x, y)}{\partial(u, v)} =
    \begin{vmatrix}
        \frac{\partial x}{\partial u} & \frac{\partial x}{\partial v} \\
        \frac{\partial y}{\partial u} & \frac{\partial y}{\partial v}
    \end{vmatrix}.
    \]
    Since \( x = u + v \) and \( y = u - v \):
    \[
    \frac{\partial x}{\partial u} = 1, \quad \frac{\partial x}{\partial v} = 1, \quad \frac{\partial y}{\partial u} = 1, \quad \frac{\partial y}{\partial v} = -1.
    \]
    Thus,
    \[
    \frac{\partial(x, y)}{\partial(u, v)} = \begin{vmatrix} 1 & 1 \\ 1 & -1 \end{vmatrix} = (1)(-1) - (1)(1) = -2.
    \]
    Therefore,
    \[
    dx \, dy = \left| \frac{\partial(x, y)}{\partial(u, v)} \right| du \, dv = 2 \, du \, dv.
    \]

    \item \textbf{Evaluate the integrals \( I_1 = \iint_{P} dA \) and \( I_2 = \iint_{P} (x^2 - y^2) \, dA \) using the change of variables.}

    \subsection*{Integral \( I_1 \)}
    Since \( P \) is the region in \( xy \)-coordinates with vertices \( (0, 0), (1, 1), (1, -1), (2, 0) \), it corresponds to the unit square \( 0 \leq u \leq 1 \), \( 0 \leq v \leq 1 \) in \( (u, v) \)-coordinates. Thus,
    \[
    I_1 = \iint_{P} dA = \int_{0}^{1} \int_{0}^{1} 2 \, du \, dv = 2 \int_{0}^{1} \int_{0}^{1} du \, dv.
    \]
    Evaluating the inner integral with respect to \( u \):
    \[
    \int_{0}^{1} \int_{0}^{1} du \, dv = \int_{0}^{1} \left[ u \right]_{0}^{1} \, dv = \int_{0}^{1} 1 \, dv = \left[ v \right]_{0}^{1} = 1.
    \]
    So,
    \[
    I_1 = 2 \cdot 1 = 2.
    \]

    \subsection*{Integral \( I_2 \)}
    To evaluate \( I_2 = \iint_{P} (x^2 - y^2) \, dA \), we express \( x^2 - y^2 \) in terms of \( u \) and \( v \). Since \( x = u + v \) and \( y = u - v \),
    \[
    x^2 - y^2 = (u + v)^2 - (u - v)^2.
    \]
    Expanding each term:
    \[
    (u + v)^2 = u^2 + 2uv + v^2 \quad \text{and} \quad (u - v)^2 = u^2 - 2uv + v^2.
    \]
    Therefore,
    \[
    x^2 - y^2 = (u^2 + 2uv + v^2) - (u^2 - 2uv + v^2) = 4uv.
    \]
    So,
    \[
    I_2 = \iint_{P} (x^2 - y^2) \, dA = \int_{0}^{1} \int_{0}^{1} 4uv \cdot 2 \, du \, dv = 8 \int_{0}^{1} \int_{0}^{1} uv \, du \, dv.
    \]
    Now, evaluate the inner integral with respect to \( u \):
    \[
    \int_{0}^{1} \int_{0}^{1} uv \, du \, dv = \int_{0}^{1} \left[ \frac{u^2}{2} \right]_{0}^{1} v \, dv = \int_{0}^{1} \frac{1}{2} v \, dv = \frac{1}{2} \int_{0}^{1} v \, dv = \frac{1}{2} \left[ \frac{v^2}{2} \right]_{0}^{1} = \frac{1}{2} \cdot \frac{1}{2} = \frac{1}{4}.
    \]
    Therefore,
    \[
    I_2 = 8 \cdot \frac{1}{4} = 2.
    \]

\end{enumerate}

\newpage

\subsection{Conclusion}

The values of the integrals are:
\[
I_1 = 2, \quad I_2 = 2.
\]






\newpage

\section{Solution}

We are asked to evaluate the integral:
\[
I = \iint_{R} x^2 y^4 \, dA
\]
where \( R \) is the region bounded by \( xy = 4 \), \( xy = 8 \), \( y = x \), and \( y = 4x \). We will use the transformation:
\[
x = 2 \sqrt{\dfrac{u}{v}}, \quad y = 2 \sqrt{u v}
\]

\newpage

\subsection{Step 1: Change of Variables}

First, we express \( u \) and \( v \) in terms of \( x \) and \( y \) to find the limits of integration and compute the Jacobian determinant.

From the given transformation:
\[
x y = \left(2 \sqrt{\dfrac{u}{v}}\right)\left(2 \sqrt{u v}\right) = 4 u
\]
Thus,
\[
u = \dfrac{x y}{4}
\]
Also,
\[
v = \dfrac{y}{x}
\]

\newpage

\subsection{Step 2: Determining the Region \( S \) in the \( uv \)-Plane}

The region \( R \) in the \( xy \)-plane is bounded by:
\begin{enumerate}
    \item \( x y = 4 \implies u = \dfrac{4}{4} = 1 \)
    \item \( x y = 8 \implies u = \dfrac{8}{4} = 2 \)
    \item \( y = x \implies v = \dfrac{x}{x} = 1 \)
    \item \( y = 4x \implies v = \dfrac{4x}{x} = 4 \)
\end{enumerate}
Therefore, in the \( uv \)-plane, the region \( S \) is defined by \( 1 \leq u \leq 2 \) and \( 1 \leq v \leq 4 \).

\newpage

\subsection{Step 3: Computing the Jacobian Determinant}

We compute the Jacobian determinant \( J(u,v) = \left| \dfrac{\partial(x,y)}{\partial(u,v)} \right| \).

First, compute the partial derivatives:
\[
\begin{aligned}
\dfrac{\partial x}{\partial u} &= 2 \left( \dfrac{u}{v} \right)^{-1/2} \cdot \dfrac{1}{2} v^{-1} = \dfrac{1}{u^{1/2} v^{1/2}} \\
\dfrac{\partial x}{\partial v} &= 2 \left( \dfrac{u}{v} \right)^{-1/2} \cdot \left( -\dfrac{u}{2 v^2} \right) = -\dfrac{u^{1/2}}{v^{3/2}} \\
\dfrac{\partial y}{\partial u} &= 2 (u v)^{-1/2} \cdot \dfrac{1}{2} v = \dfrac{v^{1/2}}{u^{1/2}} \\
\dfrac{\partial y}{\partial v} &= 2 (u v)^{-1/2} \cdot \dfrac{1}{2} u = \dfrac{u^{1/2}}{v^{1/2}}
\end{aligned}
\]
Compute the determinant:
\[
\begin{aligned}
J(u,v) &= \left| \begin{array}{cc}
\dfrac{1}{u^{1/2} v^{1/2}} & -\dfrac{u^{1/2}}{v^{3/2}} \\
\dfrac{v^{1/2}}{u^{1/2}} & \dfrac{u^{1/2}}{v^{1/2}}
\end{array} \right| \\
&= \left( \dfrac{1}{u^{1/2} v^{1/2}} \cdot \dfrac{u^{1/2}}{v^{1/2}} \right) - \left( -\dfrac{u^{1/2}}{v^{3/2}} \cdot \dfrac{v^{1/2}}{u^{1/2}} \right) \\
&= \dfrac{1}{v} + \dfrac{1}{v} = \dfrac{2}{v}
\end{aligned}
\]

\newpage

\subsection{Step 4: Transforming the Integral}

We substitute \( x \), \( y \), and \( dA \) into the integral:
\[
I = \iint_{S} x^2 y^4 \left| \dfrac{\partial(x,y)}{\partial(u,v)} \right| \, du \, dv
\]
Compute \( x^2 \) and \( y^4 \):
\[
\begin{aligned}
x^2 &= \left( 2 \sqrt{\dfrac{u}{v}} \right)^2 = 4 \left( \dfrac{u}{v} \right) \\
y^4 &= \left( 2 \sqrt{u v} \right)^4 = 16 (u v)^2 = 16 u^2 v^2 \\
\end{aligned}
\]
Thus,
\[
x^2 y^4 = 4 \left( \dfrac{u}{v} \right) \times 16 u^2 v^2 = 64 u^3 v
\]
Including the Jacobian determinant:
\[
x^2 y^4 \left| \dfrac{\partial(x,y)}{\partial(u,v)} \right| = 64 u^3 v \times \dfrac{2}{v} = 128 u^3
\]

\newpage

\subsection{Step 5: Evaluating the Integral}

The integral becomes:
\[
I = \int_{u=1}^2 \int_{v=1}^4 128 u^3 \, dv \, du
\]
Since the integrand is independent of \( v \), we can integrate with respect to \( v \):
\[
\int_{v=1}^4 128 u^3 \, dv = 128 u^3 (4 - 1) = 384 u^3
\]
Now, integrate with respect to \( u \):
\[
I = \int_{u=1}^2 384 u^3 \, du = 384 \left[ \dfrac{u^4}{4} \right]_1^2 = 384 \left( \dfrac{16}{4} - \dfrac{1}{4} \right) = 384 \left( \dfrac{15}{4} \right) = 96 \times 15 = 1440
\]

\newpage

\subsection{Conclusion}

Therefore, the value of the integral is:
\[
I = 1440
\]




\newpage

\section{Solution}

We are asked to find the area of the region in the plane defined by the inequalities:
\[
\begin{cases}
1 \leq x y \leq 3, \\
2 \leq x y^2 \leq 10, \\
x \geq 0.
\end{cases}
\]

\newpage

\subsection{Step 1: Choose an Appropriate Change of Variables}

We notice that the inequalities involve \( x y \) and \( x y^2 \). Let's define new variables to simplify the region:
\[
u = x y, \quad v = y.
\]
This substitution will help us transform the given region into a simpler one in the \( uv \)-plane.

\newpage

\subsection{Step 2: Express \( x \) and \( y \) in Terms of \( u \) and \( v \)}

From the substitution:
\[
x = \frac{u}{y} = \frac{u}{v}.
\]

\newpage

\subsection{Step 3: Compute the Jacobian Determinant}

We need to compute the Jacobian determinant \( J(u, v) \) of the transformation:
\[
J(u, v) = \left| \frac{\partial(x, y)}{\partial(u, v)} \right|.
\]

Compute the partial derivatives:
\[
\begin{aligned}
\frac{\partial x}{\partial u} &= \frac{1}{v}, \quad &\frac{\partial x}{\partial v} &= -\frac{u}{v^2}, \\
\frac{\partial y}{\partial u} &= 0, \quad &\frac{\partial y}{\partial v} &= 1.
\end{aligned}
\]

Compute the determinant:
\[
J(u, v) = \left| \begin{array}{cc}
\frac{1}{v} & -\frac{u}{v^2} \\
0 & 1 \\
\end{array} \right| = \frac{1}{v}.
\]

\newpage

\subsection{Step 4: Transform the Inequalities to the \( uv \)-Plane}

Transform the inequalities using the substitution:
\begin{itemize}
    \item \( 1 \leq x y = u \leq 3 \).
    \item \( 2 \leq x y^2 = u v \leq 10 \).
    \item \( x \geq 0 \implies \frac{u}{v} \geq 0 \).
\end{itemize}

Since \( v = y \) and \( x = \frac{u}{v} \geq 0 \), we have:
\[
\frac{u}{v} \geq 0 \implies u \text{ and } v \text{ have the same sign}.
\]
Given \( 1 \leq u \leq 3 \) (so \( u > 0 \)), it follows that \( v > 0 \).

Thus, the inequalities become:
\[
\begin{cases}
1 \leq u \leq 3, \\
2 \leq u v \leq 10, \\
v > 0.
\end{cases}
\]

\newpage

\subsection{Step 5: Determine the Limits of Integration}

For \( u \) between \( 1 \) and \( 3 \), \( v \) varies according to:
\[
2 \leq u v \leq 10 \implies \frac{2}{u} \leq v \leq \frac{10}{u}.
\]

Therefore, the limits are:
\[
\begin{aligned}
u &\in [1, 3], \\
v &\in \left[ \dfrac{2}{u}, \dfrac{10}{u} \right].
\end{aligned}
\]

\newpage

\subsection{Step 6: Set Up the Integral}

The area \( A \) is given by:
\[
A = \iint_{\text{Region}} dx \, dy = \iint_{\text{Region}} \left| \dfrac{\partial(x, y)}{\partial(u, v)} \right| du \, dv = \iint_{\text{Region}} \dfrac{1}{v} \, du \, dv.
\]

\newpage

\subsection{Step 7: Evaluate the Integral}

Compute the integral:
\[
A = \int_{u=1}^{3} \int_{v=\frac{2}{u}}^{\frac{10}{u}} \dfrac{1}{v} \, dv \, du.
\]

First, integrate with respect to \( v \):
\[
\int_{\frac{2}{u}}^{\frac{10}{u}} \dfrac{1}{v} \, dv = \ln\left( \dfrac{10}{u} \right) - \ln\left( \dfrac{2}{u} \right) = \ln\left( \dfrac{10}{u} \times \dfrac{u}{2} \right) = \ln\left( \dfrac{10}{2} \right) = \ln 5.
\]

So the integral simplifies to:
\[
A = \ln 5 \int_{u=1}^{3} du = \ln 5 \times (3 - 1) = 2 \ln 5.
\]

\newpage

\subsection{Conclusion}

The area of the region is:
\[
A = 2 \ln 5.
\]


\end{document}
