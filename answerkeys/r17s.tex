\documentclass[11pt]{article}
\usepackage{amsmath,amsthm,amssymb}
\usepackage[colorlinks]{hyperref}
\usepackage{tikz, pgfplots}
\pgfplotsset{compat=1.17}

\begin{document}
\title{Quick answer key to Recitation 16}
\author{ChatGPT 4o}
\date{4 November 2024}
\maketitle

Use the table of contents below to skip to a specific part
without seeing spoilers to the other parts.

I just used ChatGPT to write this one quickly.
ChatGPT can make mistakes, so if you spot anything that's wrong, flag me to ask.

\tableofcontents

\newpage

\section{Problem 1}

Consider the vector field $\mathbf{F} = (y^2 + 2x)\mathbf{i} + a x y \mathbf{j}$.

\newpage

\subsection{(a) Find the curl of $\mathbf{F}$.}

In two dimensions, the curl of a vector field $\mathbf{F} = P\mathbf{i} + Q\mathbf{j}$ is given by:
\[
\operatorname{curl} \mathbf{F} = \frac{\partial Q}{\partial x} - \frac{\partial P}{\partial y}
\]
Given:
\[
P = y^2 + 2x \quad \text{and} \quad Q = a x y
\]
Compute the partial derivatives:
\[
\frac{\partial Q}{\partial x} = a y \quad \text{and} \quad \frac{\partial P}{\partial y} = 2 y
\]
Therefore, the curl is:
\[
\operatorname{curl} \mathbf{F} = a y - 2 y = (a - 2) y
\]

\newpage

\subsection{(b) For what values of $a$ is $\mathbf{F}$ a conservative gradient field?}

A vector field is conservative if its curl is zero throughout the domain. Setting the curl to zero:
\[
(a - 2) y = 0
\]
For the curl to be zero for all $y$, we require:
\[
a - 2 = 0 \implies a = 2
\]

\newpage

\subsection{(c) For those values of $a$ find a potential function.}

For $a = 2$, the vector field becomes:
\[
\mathbf{F} = (y^2 + 2x)\mathbf{i} + 2 x y \mathbf{j}
\]
We seek a function $f(x, y)$ such that:
\[
\frac{\partial f}{\partial x} = y^2 + 2x \quad \text{and} \quad \frac{\partial f}{\partial y} = 2 x y
\]
Integrate $\frac{\partial f}{\partial x}$ with respect to $x$:
\[
f(x, y) = \int (y^2 + 2x) \, dx = y^2 x + x^2 + \phi(y)
\]
Differentiate $f(x, y)$ with respect to $y$:
\[
\frac{\partial f}{\partial y} = 2 y x + \phi'(y)
\]
Set equal to $\frac{\partial f}{\partial y}$:
\[
2 y x + \phi'(y) = 2 x y \implies \phi'(y) = 0
\]
Thus, $\phi(y)$ is a constant. The potential function is:
\[
f(x, y) = x y^2 + x^2 + C
\]

\newpage

\section{Problem 2}

Consider the vector field $\mathbf{F} = e^{x+y}\left( (x + a)\mathbf{i} + x \mathbf{j} \right)$.

\newpage

\subsection{(a) Find the curl of $\mathbf{F}$.}

Given:
\[
P = e^{x + y} (x + a) \quad \text{and} \quad Q = e^{x + y} x
\]
Compute the partial derivatives:
\[
\frac{\partial Q}{\partial x} = e^{x + y} x + e^{x + y} = e^{x + y} (x + 1)
\]
\[
\frac{\partial P}{\partial y} = e^{x + y} (x + a)
\]
Therefore, the curl is:
\[
\operatorname{curl} \mathbf{F} = \frac{\partial Q}{\partial x} - \frac{\partial P}{\partial y} = e^{x + y} (x + 1) - e^{x + y} (x + a) = e^{x + y} (1 - a)
\]

\newpage

\subsection{(b) For what values of $a$ is $\mathbf{F}$ a conservative gradient field?}

Set the curl to zero:
\[
e^{x + y} (1 - a) = 0
\]
Since $e^{x + y} > 0$ for all $x$, $y$, we have:
\[
1 - a = 0 \implies a = 1
\]

\newpage

\subsection{(c) For those values of $a$ find a potential function.}

For $a = 1$, the vector field simplifies to:
\[
\mathbf{F} = e^{x + y} (x + 1)\mathbf{i} + e^{x + y} x \mathbf{j}
\]
We need a function $f(x, y)$ such that:
\[
\frac{\partial f}{\partial x} = e^{x + y} (x + 1) \quad \text{and} \quad \frac{\partial f}{\partial y} = e^{x + y} x
\]
Integrate $\frac{\partial f}{\partial x}$ with respect to $x$:
\[
f(x, y) = \int e^{x + y} (x + 1) \, dx
\]
Let $u = x + y$, then $du = dx$ (since $y$ is treated as constant during integration with respect to $x$). Rewrite the integral:
\[
f(x, y) = \int e^{u} (x + 1) \, du
\]
But since $x = u - y$, we have:
\[
f(x, y) = \int e^{u} (u - y + 1) \, du
\]
Simplify and integrate:
\[
f(x, y) = \int e^{u} (u - y + 1) \, du = e^{u} (u - y)
\]
Substitute back $u = x + y$:
\[
f(x, y) = e^{x + y} (x + y - y) = e^{x + y} x
\]
Compute the partial derivatives to verify:
\[
\frac{\partial f}{\partial x} = e^{x + y} x + e^{x + y} = e^{x + y} (x + 1)
\]
\[
\frac{\partial f}{\partial y} = e^{x + y} x
\]
Thus, the potential function is:
\[
f(x, y) = x e^{x + y}
\]

\newpage

\section{Problem 3}

Consider the vector field $\mathbf{G} = -\dfrac{y}{x^2 + y^2} \mathbf{i} + \dfrac{x}{x^2 + y^2} \mathbf{j}$.

\newpage

\subsection{(a) Calculate the curl of $\mathbf{G}$.}

Given:
\[
P = -\dfrac{y}{x^2 + y^2} \quad \text{and} \quad Q = \dfrac{x}{x^2 + y^2}
\]
Compute the partial derivatives:
\[
\frac{\partial Q}{\partial x} = \frac{(1)(x^2 + y^2) - x(2x)}{(x^2 + y^2)^2} = \frac{-x^2 + y^2}{(x^2 + y^2)^2}
\]
\[
\frac{\partial P}{\partial y} = -\left( \frac{(1)(x^2 + y^2) - y(2y)}{(x^2 + y^2)^2} \right) = -\left( \frac{x^2 - y^2}{(x^2 + y^2)^2} \right)
\]
Compute the curl:
\[
\operatorname{curl} \mathbf{G} = \frac{\partial Q}{\partial x} - \frac{\partial P}{\partial y} = \frac{-x^2 + y^2}{(x^2 + y^2)^2} - \left( -\frac{x^2 - y^2}{(x^2 + y^2)^2} \right) = 0
\]
Therefore, the curl of $\mathbf{G}$ is zero everywhere except possibly at the origin.

\newpage

\subsection{(b) Show that $\mathbf{G}$ is not a gradient vector field by calculating the line integral $\displaystyle \int_{C} \mathbf{G} \cdot d\mathbf{r}$ for the closed curve $C$ given by the unit circle, oriented counterclockwise.}

Parametrize the unit circle:
\[
x = \cos \theta, \quad y = \sin \theta, \quad \theta \in [0, 2\pi]
\]
Compute differentials:
\[
dx = -\sin \theta \, d\theta, \quad dy = \cos \theta \, d\theta
\]
Evaluate $\mathbf{G}$ along $C$:
\[
P = -\frac{y}{x^2 + y^2} = -\frac{\sin \theta}{1} = -\sin \theta, \quad Q = \frac{x}{x^2 + y^2} = \frac{\cos \theta}{1} = \cos \theta
\]
Compute the dot product:
\[
\mathbf{G} \cdot d\mathbf{r} = P \, dx + Q \, dy = (-\sin \theta)(-\sin \theta \, d\theta) + (\cos \theta)(\cos \theta \, d\theta) = (\sin^2 \theta + \cos^2 \theta) \, d\theta = d\theta
\]
Integrate over $C$:
\[
\int_{C} \mathbf{G} \cdot d\mathbf{r} = \int_{0}^{2\pi} d\theta = 2\pi
\]
Since the line integral around the closed curve $C$ is non-zero, $\mathbf{G}$ is not a conservative vector field.


\end{document}
