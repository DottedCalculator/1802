\documentclass[addpoints, 12pt,answers]{exam}
\usepackage{fullpage, tikz, multicol, amsmath, amssymb}
\usepackage{graphpap, color}
\usepackage{vmargin}


\begin{document}
\begin{center}
{\bf 18.02, Fall 2024, Recitation 5}
\end{center}

\vspace{5 mm}

\begin{questions}


\question Compute the eigenvalues and eigenvectors of the following matrices:
\begin{itemize}
\item $A = \begin{pmatrix} 1 & 2\\2 & 1\end{pmatrix}$
\item $B =  \begin{pmatrix} 1 & 2\\-1 & -2\end{pmatrix}$
\end{itemize}
\begin{solution}
\subsubsection*{For matrix $A$}
We first get the characteristic polynomial of $A$.
Since this is a $2 \times 2$ matrix (Poonen Theorem 6.8), it's easiest to just note that
$\operatorname{Tr} A = 1+1 = 2$ and $\det A = 1 \cdot 1 - 2 \cdot 2 = -3$ to get
\[ \lambda^2 - 2 \lambda - 3. \]
Of course, if you like you can also do the long way to get the same result:
\begin{align*}
  \det(A - \lambda I) &= \det\begin{pmatrix} 1 - \lambda & 2 \\ 2 & 1 - \lambda \end{pmatrix} = (1 - \lambda)(1 - \lambda) - 2 \cdot 2 \\
  &= (1 - \lambda)^2 - 4 = 1 - 2\lambda + \lambda^2 - 4 = \lambda^2 - 2\lambda - 3.
\end{align*}
Anyway, the two roots are $\lambda_1 = 3$ and $\lambda_2 = -1$.

For each eigenvalue, we solve the system $(A - \lambda I)\mathbf{v} = 0$ where $\mathbf{v} = \begin{pmatrix} v \\ w \end{pmatrix}$.
\begin{itemize}
\item Eigenvalue $\lambda_1 = 3$:
We solve:
\[ (A - 3I) \mathbf{v} = \begin{pmatrix} 1 - 3 & 2 \\ 2 & 1 - 3 \end{pmatrix} \begin{pmatrix} v \\ w \end{pmatrix} = \begin{pmatrix} -2 & 2 \\ 2 & -2 \end{pmatrix} \begin{pmatrix} v \\ w \end{pmatrix} = \begin{pmatrix} 0 \\ 0 \end{pmatrix} \]
This gives the system:
\[ -2v + 2w = 0, \quad 2v - 2w = 0 \]
Both equations reduce to:
\[ v = w \]
Thus, the eigenvectors corresponding to $\lambda_1 = 3$ are all the multiples of:
\[ \mathbf{v}_1 = \begin{pmatrix} 1 \\ 1 \end{pmatrix}. \]

\item {Eigenvalue $\lambda_2 = -1$:}
We solve:
\[ (A + I) \mathbf{v} = \begin{pmatrix} 1 + 1 & 2 \\ 2 & 1 + 1 \end{pmatrix} \begin{pmatrix} v \\ w \end{pmatrix} = \begin{pmatrix} 2 & 2 \\ 2 & 2 \end{pmatrix} \begin{pmatrix} v \\ w \end{pmatrix} = \begin{pmatrix} 0 \\ 0 \end{pmatrix} \]
This gives the system:
\[ 2v + 2w = 0, \quad 2v + 2w = 0 \]
Both equations reduce to:
\[ v = -w \]
Thus, the eigenvectors corresponding to $\lambda_1 = 3$ are all the multiples of:
\[ \mathbf{v}_2 = \begin{pmatrix} 1 \\ -1 \end{pmatrix}. \]
\end{itemize}

\subsubsection*{For matrix $B$}
Now we do the same for $B$.
Since $\operatorname{Tr} B = 1 + (-2) = -1$ and $\det B = 1 \cdot (-2) - 2 \cdot (-1) = 0$
the characteristic polynomial is given by
\[ \lambda^2 - \lambda = 0. \]
The two eigenvectors are therefore
\[ \lambda_1 = 0, \quad \lambda_2 = -1. \]

For each eigenvalue, we solve $(B - \lambda I)\mathbf{v} = 0$.
\begin{itemize}
\item {Eigenvalue $\lambda_1 = 0$:}
We solve:
\[ B \mathbf{v} = \begin{pmatrix} 1 & 2 \\ -1 & -2 \end{pmatrix} \begin{pmatrix} v \\ w \end{pmatrix} = \begin{pmatrix} 0 \\ 0 \end{pmatrix}. \]
This gives the system:
\[ v + 2w = 0, \quad -v - 2w = 0. \]
Both equations reduce to:
\[ v = -2w. \]
Thus, the eigenvectors corresponding to $\lambda_1 = 0$ are all the multiples of:
\[ \mathbf{v}_1 = \begin{pmatrix} -2 \\ 1 \end{pmatrix}. \]

\item {Eigenvalue $\lambda_2 = -1$:}
We solve:
\[ (B + I) \mathbf{v} = \begin{pmatrix} 1 + 1 & 2 \\ -1 & -2 + 1 \end{pmatrix} \begin{pmatrix} v \\ w \end{pmatrix} = \begin{pmatrix} 2 & 2 \\ -1 & -1 \end{pmatrix} \begin{pmatrix} v \\ w \end{pmatrix} = \begin{pmatrix} 0 \\ 0 \end{pmatrix}. \]
This gives the system:
\[ 2v + 2w = 0, \quad -v - w = 0. \]
Both equations reduce to:
\[ v = -w. \]
Thus, the eigenvectors corresponding to $\lambda_2 = -1$ are all the scalar multiples of
\[ \mathbf{v}_2 = \begin{pmatrix} 1 \\ -1 \end{pmatrix}. \]
\end{itemize}
\end{solution}

\question Consider the matrix  $C = \begin{pmatrix} -1 & 0 & 1\\-3 &4& 1\\0  &0& 2\end{pmatrix}$.
\begin{itemize}
\item What are the eigenvalues of $C$?
\item What are the corresponding eigenvectors (up to scalar multiple)?
\end{itemize}
\begin{solution}
  This is a $3 \times 3$ matrix, so we have to actually calculate the characteristic polynomial.
  The characteristic polynomial is given by:
  \[ \det(C - \lambda I) = 0 \]
  where $I$ is the identity matrix and $\lambda$ is the eigenvalue. We compute $C - \lambda I$:
  \[
  C - \lambda I = \begin{pmatrix} -1 & 0 & 1 \\ -3 & 4 & 1 \\ 0 & 0 & 2 \end{pmatrix} - \lambda \begin{pmatrix} 1 & 0 & 0 \\ 0 & 1 & 0 \\ 0 & 0 & 1 \end{pmatrix} = \begin{pmatrix} -1 - \lambda & 0 & 1 \\ -3 & 4 - \lambda & 1 \\ 0 & 0 & 2 - \lambda \end{pmatrix}.
  \]

  Now, we compute the determinant:
  \[
  \det(C - \lambda I) = \det \begin{pmatrix} -1 - \lambda & 0 & 1 \\ -3 & 4 - \lambda & 1 \\ 0 & 0 & 2 - \lambda \end{pmatrix}.
  \]
  We expand this determinant along the third row:
  \[
  \det(C - \lambda I) = (2 - \lambda) \cdot \det \begin{pmatrix} -1 - \lambda & 0 \\ -3 & 4 - \lambda \end{pmatrix}.
  \]
  Now, we compute the $2 \times 2$ determinant:
  \[
  \det \begin{pmatrix} -1 - \lambda & 0 \\ -3 & 4 - \lambda \end{pmatrix} = (-1 - \lambda)(4 - \lambda) - (0)(-3) = (\lambda+1)(\lambda-4).
  \]
  Hence the characteristic polynomial is exactly
  \[ (\lambda+1)(\lambda-4)(2-\lambda) = 0. \]
  The eigenvalues are hence given by
  \begin{align*}
    \lambda_1 &= 2 \\
    \lambda_2 &= 4 \\
    \lambda_3 &= -1.
  \end{align*}
  For each eigenvalue, we solve the system $(C - \lambda I)\mathbf{v} = 0$, where $\mathbf{v} = \begin{pmatrix} v_1 \\ v_2 \\ v_3 \end{pmatrix}$.
  \begin{itemize}
  \item {Eigenvalue $\lambda_1 = 2$:}
  We solve:
  \begin{align*}
    (C - 2I) \mathbf{v}
    &= \begin{pmatrix} -1 - 2 & 0 & 1 \\ -3 & 4 - 2 & 1 \\ 0 & 0 & 2 - 2 \end{pmatrix} \begin{pmatrix} v_1 \\ v_2 \\ v_3 \end{pmatrix} \\
    &= \begin{pmatrix} -3 & 0 & 1 \\ -3 & 2 & 1 \\ 0 & 0 & 0 \end{pmatrix}
    \begin{pmatrix} v_1 \\ v_2 \\ v_3 \end{pmatrix} = \begin{pmatrix} 0 \\ 0 \\ 0 \end{pmatrix}.
  \end{align*}
  This gives the system of equations:
  \[ -3v_1 + v_3 = 0, \qquad  -3v_1 + 2v_2 + v_3 = 0. \]
  From the first equation, we get $v_3 = 3v_1$.
  Substituting this into the second equation:
  \[ -3v_1 + 2v_2 + 3v_1 = 0 \implies 2v_2 = 0 \implies v_2 = 0. \]
  Thus, the eigenvector corresponding to $\lambda_1 = 2$ is (up to constant multiple):
  \[ \mathbf{v}_1 = \begin{pmatrix} 1 \\ 0 \\ 3 \end{pmatrix}. \]

  \item {Eigenvalue $\lambda_2 = 4$:}
  We solve:
  \begin{align*}
    (C - 4I) \mathbf{v}
    &= \begin{pmatrix} -1 - 4 & 0 & 1 \\ -3 & 4 - 4 & 1 \\ 0 & 0 & 2 - 4 \end{pmatrix} \begin{pmatrix} v_1 \\ v_2 \\ v_3 \end{pmatrix} \\
    &= \begin{pmatrix} -5 & 0 & 1 \\ -3 & 0 & 1 \\ 0 & 0 & -2 \end{pmatrix} \begin{pmatrix} v_1 \\ v_2 \\ v_3 \end{pmatrix} = \begin{pmatrix} 0 \\ 0 \\ 0 \end{pmatrix}.
  \end{align*}
  This gives the system of equations:
  \[ -5v_1 + v_3 = 0 , \qquad -3v_1 + v_3 = 0, \qquad -2v_3 = 0. \]
  From the third equation, we get $v_3 = 0$.
  Substituting $v_3 = 0$ into the first and second equations gives $v_1 = 0$.
  Therefore, $v_2$ is free to vary.

  Thus, the eigenvector corresponding to $\lambda_2 = 4$ is up to constant multiple:
  \[ \mathbf{v}_2 = \begin{pmatrix} 0 \\ 1 \\ 0 \end{pmatrix}. \]

  \item {Eigenvalue $\lambda_3 = -1$:}
  We solve:
  \begin{align*}
    (C + I) \mathbf{v}
    &= \begin{pmatrix} -1 + 1 & 0 & 1 \\ -3 & 4 + 1 & 1 \\ 0 & 0 & 2 + 1 \end{pmatrix} \begin{pmatrix} v_1 \\ v_2 \\ v_3 \end{pmatrix} \\
    &= \begin{pmatrix} 0 & 0 & 1 \\ -3 & 5 & 1 \\ 0 & 0 & 3 \end{pmatrix} \begin{pmatrix} v_1 \\ v_2 \\ v_3 \end{pmatrix} = \begin{pmatrix} 0 \\ 0 \\ 0 \end{pmatrix}.
  \end{align*}
  This gives the system of equations:
  \[ v_3 = 0, \qquad -3v_1 + 5v_2 = 0, \qquad 3v_3 = 0. \]
  From the second equation, we get $v_1 = \frac{5}{3}v_2$.
  Thus, the eigenvector corresponding to $\lambda_3 = -1$ is (up to constant multiple):
  \[ \mathbf{v}_3 = \begin{pmatrix} 5 \\ 3 \\ 0 \end{pmatrix}. \]
  \end{itemize}
\end{solution}


\question For any real numbers $a$ and $b$, compute eigenvalues and eigenvectors of the following matrices.
\begin{itemize}
\item $A =  \begin{pmatrix} a & 0\\0 & b\end{pmatrix}$

\item $B =  \begin{pmatrix} a & 1\\0 & a\end{pmatrix}$.
\end{itemize}
When $a=b$, what is the difference between the answers for $A$ and $B$?

\begin{solution}
  For the matrix $A$, one could follow the brute-force procedure in problem 1 again.
  But it might be easier to find the eigenvectors by inspection:
  \begin{itemize}
    \item $A \begin{pmatrix} 1 \\ 0 \end{pmatrix} = \begin{pmatrix} a \\ 0 \end{pmatrix}$;
    hence $\mathbf{e}_1$ (and its multiples) are eigenvectors with eigenvalue $a$.
    \item $A \begin{pmatrix} 0 \\ 1 \end{pmatrix} = \begin{pmatrix} 0 \\ b \end{pmatrix}$;
    hence $\mathbf{e}_1$ (and its multiples) are eigenvectors with eigenvalue $b$.
  \end{itemize}
  When $a \neq b$, this means $A$ has eigenvalues $a$ and $b$
  with multiples of the basis vectors as eigenvectors.
  (Indeed, this checks out with $\operatorname{Tr} A = a+b$ and $\det A = ab$.)
  When $a = b$, however, the matrix $A = aI$ has \emph{every} vector of $\mathbb R^2$
  as an eigenvector of eigenvalue $a$; see Remark 6.13 of Poonen.

  For the matrix $B$, because $\operatorname{Tr} B = 2a$ and $\det A = a^2$,
  the characteristic polynomial is
  \[ \lambda^2 - 2a \lambda + a^2 = (\lambda-a)^2. \]
  Hence the only eigenvalue is $a$.
  We will show, however, there is only a single eigenline.
  Indeed, set $\begin{pmatrix} v \\ w \end{pmatrix}$ as an eigenvector.
  We have
  \[
    (B - aI) \begin{pmatrix} v \\ w \end{pmatrix}
    = \begin{pmatrix} 0 & 1 \\ 0 & 0 \end{pmatrix}
    \begin{pmatrix} v \\ w \end{pmatrix}
    = \begin{pmatrix} w \\ 0 \end{pmatrix}.
  \]
  Hence, the eigenvalues for $B$ are exactly those with $w = 0$,
  i.e.\ the multiples of the basis vector $\mathbf{e}_1 = \begin{pmatrix} 1 \\ 0 \end{pmatrix}$.

  To summarize the answer in a table:
  \begin{center}
    \begin{tabular}{ccc}
      Problem & Eigenvalues & Eigenvectors \\ \hline
      Matrix $A$, when $a \neq b$ & $a$ and $b$ & Multiples of $\mathbf{e}_1$ (for $a$) \\
      && Multiples of $\mathbf{e}_2$ (for $b$) \\
      Matrix $A$, when $a = b$ & $a$ & Every vector in $\mathbb{R}^2$ \\
      Matrix $B$ & $a$ & Multiples of $\mathbf{e}_1$ only.
    \end{tabular}
  \end{center}

  (This can feel wrong, because $B$ seems like it is ``missing''
  one dimension of eigenvectors.
  This phenomenon only happens when the characteristic polynomial has repeated roots.
  If you are curious to see what happens in higher dimensions,
  the correct keywords to search for are ``Jordan normal form''.)
\end{solution}

\end{questions}

\end{document}
