\documentclass[11pt]{article}
\usepackage{amsmath,amsthm,amssymb}
\usepackage[colorlinks]{hyperref}
\usepackage{tikz, pgfplots}
\pgfplotsset{compat=1.17}

\begin{document}
\title{Quick answer key to Recitation 21}
\author{ChatGPT 4o}
\date{December 1, 2024}
\maketitle

Use the table of contents below to skip to a specific part
without seeing spoilers to the other parts.

I just used ChatGPT to write this one quickly.
ChatGPT can make mistakes, so if you spot anything that's wrong, flag me to ask.

\tableofcontents



\newpage

\section{Flux of the Vector Field Through a Conical Surface}

We are tasked with computing the upward flux of the vector field \( \mathbf{V}(x, y, z) = \langle x, y, 2z \rangle \) through the surface \( S \), which is the portion of the cone \( z = \sqrt{x^2 + y^2} \) lying in the region \( 1 \leq z \leq 2 \).

\newpage

\subsection{Understanding the Surface \( S \)}

\begin{itemize}
    \item **Equation of the Cone:** \( z = \sqrt{x^2 + y^2} \) or equivalently \( z^2 = x^2 + y^2 \).
    \item **Bounds:** The region of interest lies between \( z = 1 \) and \( z = 2 \).
    \item **Intersection with Planes:** At \( z = 1 \), the cone intersects the plane \( z = 1 \) at \( x^2 + y^2 = 1 \). At \( z = 2 \), it intersects at \( x^2 + y^2 = 4 \).
\end{itemize}

\newpage

\subsection{Parametrization of the Surface \( S \)}

We will use cylindrical coordinates to parametrize the surface \( S \).

\subsubsection*{Cylindrical Coordinates Overview}

In cylindrical coordinates \((r, \theta, z)\):
\[
x = r \cos\theta, \quad y = r \sin\theta, \quad z = z
\]
where:
\begin{itemize}
    \item \( r \geq 0 \) is the radial distance,
    \item \( 0 \leq \theta < 2\pi \) is the azimuthal angle,
    \item \( z \) is the height.
\end{itemize}

\subsubsection*{Parametrization}

Given the cone \( z = r \), the surface \( S \) can be parametrized as:
\[
\mathbf{r}(r, \theta) = \langle r \cos\theta, \, r \sin\theta, \, r \rangle
\]
with the bounds:
\[
1 \leq r \leq 2, \quad 0 \leq \theta < 2\pi
\]

\newpage

\subsection{Computing the Tangent Vectors}

To find the flux, we need the upward-pointing normal vector. We obtain this by computing the cross product of the tangent vectors.

\paragraph{Partial Derivative with respect to \( r \):}
\[
\mathbf{r}_r = \frac{\partial \mathbf{r}}{\partial r} = \langle \cos\theta, \, \sin\theta, \, 1 \rangle
\]

\paragraph{Partial Derivative with respect to \( \theta \):}
\[
\mathbf{r}_\theta = \frac{\partial \mathbf{r}}{\partial \theta} = \langle -r \sin\theta, \, r \cos\theta, \, 0 \rangle
\]

\paragraph{Cross Product \( \mathbf{r}_r \times \mathbf{r}_\theta \):}
\[
\mathbf{r}_r \times \mathbf{r}_\theta =
\begin{vmatrix}
\mathbf{i} & \mathbf{j} & \mathbf{k} \\
\cos\theta & \sin\theta & 1 \\
- r \sin\theta & r \cos\theta & 0 \\
\end{vmatrix}
= \langle -r \cos\theta, \, -r \sin\theta, \, r \rangle
\]
This vector points upward as the \( z \)-component is positive.

\newpage

\subsection{Calculating the Flux}

The flux \( \Phi \) of \( \mathbf{V} \) through \( S \) is given by:
\[
\Phi = \iint_{S} \mathbf{V} \cdot \mathbf{n} \, dS
\]
where \( \mathbf{n} \, dS = \mathbf{r}_r \times \mathbf{r}_\theta \, dr \, d\theta = \langle -r \cos\theta, \, -r \sin\theta, \, r \rangle \, dr \, d\theta \).

\subsubsection*{Dot Product \( \mathbf{V} \cdot \mathbf{n} \)}

Evaluate \( \mathbf{V} \) at \( \mathbf{r}(r, \theta) \):
\[
\mathbf{V}(x, y, z) = \langle x, y, 2z \rangle = \langle r \cos\theta, \, r \sin\theta, \, 2r \rangle
\]
Dot product with \( \mathbf{n} \):
\[
\mathbf{V} \cdot \mathbf{n} = \langle r \cos\theta, \, r \sin\theta, \, 2r \rangle \cdot \langle -r \cos\theta, \, -r \sin\theta, \, r \rangle \] \[= -r^2 \cos^2\theta - r^2 \sin^2\theta + 2r^2 = r^2 ( -\cos^2\theta - \sin^2\theta + 2 ) \] \[ = r^2 ( -1 + 2 ) = r^2
\]
Thus, the integrand simplifies to \( r^2 \).

\subsubsection*{Setting Up the Integral}

The flux integral becomes:
\[
\Phi = \int_{0}^{2\pi} \int_{1}^{2} r^2 \, dr \, d\theta
\]

\subsubsection*{Evaluating the Integral}

\paragraph{Integrate with respect to \( r \):}
\[
\int_{1}^{2} r^2 \, dr = \left[ \frac{r^3}{3} \right]_{1}^{2} = \frac{8}{3} - \frac{1}{3} = \frac{7}{3}
\]

\paragraph{Integrate with respect to \( \theta \):}
\[
\int_{0}^{2\pi} \frac{7}{3} \, d\theta = \frac{7}{3} \times 2\pi = \frac{14\pi}{3}
\]

\newpage

\subsection{Final Answer}

The upward flux of \( \mathbf{V} = \langle x, y, 2z \rangle \) through the surface \( S \) is:
\[
\boxed{ \Phi = \frac{14\pi}{3} }
\]

\newpage

\section{Link to second question}
Supplementary material V9, example 1, on page 2.

\url{https://ocw.mit.edu/courses/18-02-multivariable-calculus-fall-2007/b99f7b21fccbe519c0839995574b4da8_surface_integrls.pdf}

\newpage

\section{Link to third question}
Supplementary material V9, example 2, on page 3.

\url{https://ocw.mit.edu/courses/18-02-multivariable-calculus-fall-2007/b99f7b21fccbe519c0839995574b4da8_surface_integrls.pdf}



\newpage

\section{Flux of the Vector Field Through a Cylindrical Surface}

We are tasked with computing the outward flux of the vector field
\[
\mathbf{V}(x, y, z) = \langle x^3 z^2 + y^2 z, \ x^2 y z^2 - x y z, \ x z^4 - y^5 \rangle
\]
through the surface \( S \), which is the portion of the cylinder \( x^2 + y^2 = 1 \) in the first octant (\( x, y, z \geq 0 \)) that lies below \( z = 1 \).

\newpage

\subsection{Understanding the Surface \( S \)}

\begin{itemize}
    \item **Equation of the Cylinder:** \( x^2 + y^2 = 1 \)
    \item **Bounds:**
    \[
    0 \leq z \leq 1
    \]
    \item **Octant Restriction:** \( x, y, z \geq 0 \) implies:
    \[
    0 \leq \theta \leq \frac{\pi}{2} \quad \text{(in cylindrical coordinates)}
    \]
\end{itemize}

\newpage

\subsection{Parametrization of the Surface \( S \)}

We will use cylindrical coordinates \((r, \theta, z)\) to parametrize the surface \( S \).

\subsubsection*{Cylindrical Coordinates Overview}

In cylindrical coordinates:
\[
x = r \cos\theta, \quad y = r \sin\theta, \quad z = z
\]
where:
\begin{itemize}
    \item \( r \geq 0 \) is the radial distance,
    \item \( 0 \leq \theta \leq \frac{\pi}{2} \) (first octant),
    \item \( 0 \leq z \leq 1 \).
\end{itemize}

\subsubsection*{Parametrization}

Since the cylinder \( x^2 + y^2 = 1 \) corresponds to \( r = 1 \) in cylindrical coordinates, the parametrization of \( S \) is:
\[
\mathbf{r}(\theta, z) = \langle \cos\theta, \ \sin\theta, \ z \rangle
\]
with:
\[
0 \leq \theta \leq \frac{\pi}{2}, \quad 0 \leq z \leq 1
\]

\newpage

\subsection{Computing the Tangent Vectors}

To compute the flux, we need the outward-pointing unit normal vector. We obtain this by computing the cross product of the tangent vectors.

\paragraph{Partial Derivative with respect to \( \theta \):}
\[
\mathbf{r}_\theta = \frac{\partial \mathbf{r}}{\partial \theta} = \langle -\sin\theta, \ \cos\theta, \ 0 \rangle
\]

\paragraph{Partial Derivative with respect to \( z \):}
\[
\mathbf{r}_z = \frac{\partial \mathbf{r}}{\partial z} = \langle 0, \ 0, \ 1 \rangle
\]

\paragraph{Cross Product \( \mathbf{r}_\theta \times \mathbf{r}_z \):}
\[
\mathbf{r}_\theta \times \mathbf{r}_z =
\begin{vmatrix}
\mathbf{i} & \mathbf{j} & \mathbf{k} \\
-\sin\theta & \cos\theta & 0 \\
0 & 0 & 1 \\
\end{vmatrix}
= \mathbf{i} (\cos\theta \cdot 1 - 0 \cdot 0) - \mathbf{j} (-\sin\theta \cdot 1 - 0 \cdot 0) + \mathbf{k} (-\sin\theta \cdot 0 - \cos\theta \cdot 0)
\]
\[
= \cos\theta \, \mathbf{i} + \sin\theta \, \mathbf{j} + 0 \, \mathbf{k} = \langle \cos\theta, \ \sin\theta, \ 0 \rangle
\]
This vector points outward from the cylinder.

\newpage

\subsection{Calculating the Flux}

The outward flux \( \Phi \) of \( \mathbf{V} \) through \( S \) is given by:
\[
\Phi = \iint_{S} \mathbf{V} \cdot \mathbf{n} \, dS
\]
where \( \mathbf{n} \, dS = \mathbf{r}_\theta \times \mathbf{r}_z \, d\theta \, dz = \langle \cos\theta, \ \sin\theta, \ 0 \rangle \, d\theta \, dz \).

\subsubsection*{Expressing \( \mathbf{V} \) in Cylindrical Coordinates}

At a point on \( S \), \( x = \cos\theta \), \( y = \sin\theta \), and \( z = z \). Thus, the vector field \( \mathbf{V} \) becomes:
\[
\mathbf{V}(x, y, z) = \langle (\cos\theta)^3 z^2 + (\sin\theta)^2 z, \ (\cos\theta)^2 \sin\theta z^2 - \cos\theta \sin\theta z, \ \cos\theta z^4 - (\sin\theta)^5 \rangle
\]

\subsubsection*{Dot Product \( \mathbf{V} \cdot \mathbf{n} \)}

Compute \( \mathbf{V} \cdot \mathbf{n} \):
\[
\mathbf{V} \cdot \mathbf{n} = \langle (\cos\theta)^3 z^2 + (\sin\theta)^2 z, \ (\cos\theta)^2 \sin\theta z^2 - \cos\theta \sin\theta z, \ \cos\theta z^4 - (\sin\theta)^5 \rangle \cdot \langle \cos\theta, \ \sin\theta, \ 0 \rangle
\]
\[
= \left( (\cos\theta)^3 z^2 + (\sin\theta)^2 z \right) \cos\theta + \left( (\cos\theta)^2 \sin\theta z^2 - \cos\theta \sin\theta z \right) \sin\theta + \left( \cos\theta z^4 - (\sin\theta)^5 \right) \cdot 0
\]
\[
= (\cos^4\theta z^2 + \cos\theta \sin^2\theta z) + (\cos^2\theta \sin^2\theta z^2 - \cos\theta \sin^2\theta z) + 0
\]
\[
= \cos^4\theta z^2 + \cos\theta \sin^2\theta z + \cos^2\theta \sin^2\theta z^2 - \cos\theta \sin^2\theta z
\]
\[
= \cos^4\theta z^2 + \cos^2\theta \sin^2\theta z^2
\]
\[
= (\cos^4\theta + \cos^2\theta \sin^2\theta) z^2
\]
\[
= \cos^2\theta (\cos^2\theta + \sin^2\theta) z^2
\]
\[
= \cos^2\theta \cdot 1 \cdot z^2 = \cos^2\theta z^2
\]

\subsubsection*{Setting Up the Integral}

The flux integral becomes:
\[
\Phi = \int_{0}^{\frac{\pi}{2}} \int_{0}^{1} \cos^2\theta z^2 \, dz \, d\theta
\]

\subsubsection*{Evaluating the Integral}

1. **Integrate with respect to \( z \):**
\[
\int_{0}^{1} z^2 \, dz = \left[ \frac{z^3}{3} \right]_{0}^{1} = \frac{1}{3}
\]
Thus,
\[
\Phi = \int_{0}^{\frac{\pi}{2}} \cos^2\theta \cdot \frac{1}{3} \, d\theta = \frac{1}{3} \int_{0}^{\frac{\pi}{2}} \cos^2\theta \, d\theta
\]

2. **Integrate with respect to \( \theta \):**

Recall the identity:
\[
\cos^2\theta = \frac{1 + \cos(2\theta)}{2}
\]
Thus,
\[
\int_{0}^{\frac{\pi}{2}} \cos^2\theta \, d\theta = \int_{0}^{\frac{\pi}{2}} \frac{1 + \cos(2\theta)}{2} \, d\theta = \frac{1}{2} \int_{0}^{\frac{\pi}{2}} 1 \, d\theta + \frac{1}{2} \int_{0}^{\frac{\pi}{2}} \cos(2\theta) \, d\theta
\]
\[
= \frac{1}{2} \left[ \theta \right]_{0}^{\frac{\pi}{2}} + \frac{1}{2} \left[ \frac{\sin(2\theta)}{2} \right]_{0}^{\frac{\pi}{2}}
\]
\[
= \frac{1}{2} \left( \frac{\pi}{2} - 0 \right ) + \frac{1}{4} \left( \sin\pi - \sin0 \right ) = \frac{\pi}{4} + 0 = \frac{\pi}{4}
\]
Thus,
\[
\Phi = \frac{1}{3} \times \frac{\pi}{4} = \frac{\pi}{12}
\]

\newpage

\subsection{Final Answer}

The outward flux of the vector field \( \mathbf{V} = \langle x^3 z^2 + y^2 z, \ x^2 y z^2 - x y z, \ x z^4 - y^5 \rangle \) through the surface \( S \) is:
\[
\boxed{ \Phi = \frac{\pi}{12} }
\]


\end{document}
