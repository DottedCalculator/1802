\documentclass[11pt]{amsart}
\begin{document}

\title{Exam Review (Recitation 7)}
\date{25 September 2024}
\maketitle

\section{Problems from official course}
\subsection{Vectors}
Let $P = (1,0,1)$, $Q = (1,1,2)$, and $R = (-1,1,1).$
\begin{enumerate}
\item What is the vector connecting $P$ to the \emph{midpoint} of the line segment connecting $Q$ and $R$?
\item What is the area of the triangle with vertices $P,Q,R$?
\item What is the equation of the plane through these three points?
\end{enumerate}

\subsection{Quadratic}
Suppose we have real numbers $x_1, x_2, x_3, y_1, y_2, y_3$ satisfying the conditions
$x_1^2 + x_2^2 + x_3^2 = 4$ and $y_1^2 + y_2^2 + y_3^2 = 9$.  What is the range of possible values for
$$x_1y_1 + x_2y_2+x_3y_3?$$

\subsection{Planes}
Let $P_1$ be the plane with equation $x + 2y + 3z = 0$ and $P_2$ be the plane with equation $2y - z = 0$.
\begin{enumerate}
\item Write down a vector {\bf parallel} to both $P_1$ and $P_2$.
\item Find the distance from the point $(2, 1, 4)$ to the plane $P_1$.
\end{enumerate}

\subsection{Matrix}
\begin{enumerate}
\item Calculate the matrix $M$ associated to the linear transformation of ${\bf R}^2$ given by rotation {\bf counterclockwise} by $5\pi/4$.
\item Let $N = \begin{pmatrix} 1 & 2 & 4\\ -3 & 6 & 2\end{pmatrix}.$  Calculate (if defined) the matrix products $MN$ and $NM$.
\end{enumerate}

\subsection{Parallellopiped}
Consider the vectors
\[ {\bf v}_1 = \begin{pmatrix} 2 \\ 3 \\ 0 \end{pmatrix},  {\bf v}_2 = \begin{pmatrix} 0 \\ 5 \\ 1 \end{pmatrix},  {\bf v}_3 = \begin{pmatrix} 1 \\ 2 \\ 0 \end{pmatrix}.\]

\begin{enumerate}
\item What is the volume of the parallelopiped whose sides are given by the vectors ${\bf v}_1, {\bf v}_2, {\bf v}_3$?
\item Are these vectors a basis for ${\bf R}^3$?
\end{enumerate}

\subsection{System}
Consider the following system of equations with unknown variables $x,y$, which depend on a real parameter $a$
$$
x + 3y = 0$$
$$-ax - y = 1$$

\begin{enumerate}
\item Write this system of equations as a single matrix equation with a vector unknown.
\item Find all values of $a$ for which this system has a unique solution.
\item By computing the inverse of a $2\times 2$ matrix, find the solution to this equation in terms of $a$.
\end{enumerate}

\subsection{Eigenvectors}
Consider the matrix
$$
A = \begin{pmatrix} 5 & 8 \\ 7 & 4 \end{pmatrix}.
$$

\begin{enumerate}
\item Calculate the characteristic polynomial and the eigenvalues of $A$.
\item For the largest eigenvalue, find a corresponding eigenvector ${\bf v}$.
\end{enumerate}


\subsection{Complex}
\begin{enumerate}
\item Calculate $(1 - i\sqrt{3})^7$.
\item Let $z = 2+3i$ and $w = 1+ 2i$.  Find $zw$, $z/\overline{w}$.
\end{enumerate}

\section{Older problems from Evan}
\begin{enumerate}
\item
In \({\mathbb{R}}^{3}\), compute the projection of the vector
\(\begin{pmatrix}
4 \\
5 \\
6
\end{pmatrix}\) onto the plane \(x + y + 2z = 0\).
\item
Suppose \(A\), \(B\), \(C\), \(D\) are points in \({\mathbb{R}}^{3}\).
Give a geometric interpretation for this expression:
\[|\overset{\rightarrow}{DA} \cdot \left( \overset{\rightarrow}{DB} \times \overset{\rightarrow}{DC} \right)|.\]
\item
Fix a plane \(\mathcal{P}\) in \({\mathbb{R}}^{3}\) which passes through
the origin. Consider the linear transformation
\(f:{\mathbb{R}}^{3} \rightarrow {\mathbb{R}}^{3}\) where
\(f\left( \mathbf{\mathrm{v}} \right)\) is the projection of
\(\mathbf{\mathrm{v}}\) onto \(\mathcal{P}\). Let \(M\) denote the
\(3 \times 3\) matrix associated to \(f\). Compute the determinant of
\(M\).
\item
Let \(\mathbf{\mathrm{a}}\) and \(\mathbf{\mathrm{b}}\) be two
perpendicular unit vectors in \({\mathbb{R}}^{3}\). A third vector
\(\mathbf{\mathrm{v}}\) in \({\mathbb{R}}^{3}\) lies in the span of
\(\mathbf{\mathrm{a}}\) and \(\mathbf{\mathrm{b}}\). Given that
\(\mathbf{\mathrm{v}} \cdot \mathbf{\mathrm{a}} = 2\) and
\(\mathbf{\mathrm{v}} \cdot \mathbf{\mathrm{b}} = 3\), compute the
magnitudes of the cross products
\(\mathbf{\mathrm{v}} \times \mathbf{\mathrm{a}}\) and
\(\mathbf{\mathrm{v}} \times \mathbf{\mathrm{b}}\).
\item
Compute the trace of the \(2 \times 2\) matrix \(M\) given the two
equations \[M\begin{pmatrix}
4 \\
7
\end{pmatrix} = \begin{pmatrix}
5 \\
9
\end{pmatrix}\text{  and  }M\begin{pmatrix}
5 \\
9
\end{pmatrix} = \begin{pmatrix}
4 \\
7
\end{pmatrix}.\]
\item
There are three complex numbers \(z\) satisfying \(z^{3} = 5 + 6i\).
Suppose we plot these three numbers in the complex plane. Compute the
area of the triangle they enclose.
\end{enumerate}

\end{document}
